\begin{abstract}

\noindent
{\bf Context}: Since the United States National Institute of Standards and Technology proposed Role-Based Access Control (RBAC) as documented in the RBAC Reference Model in the late 1990's, domain-targeted extensions have been proposed.  For example, the mobile domain has identified a need for permission granting and role activation.
\\
\\
\noindent
{\bf Objective}: The goal of our work is to aid practitioners and researchers in choosing an RBAC extension model, and in understanding how RBAC extension models are evaluated by providing an assessment of the state of RBAC extension models.
\\
\\
\noindent
{\bf Method}: We performed a systematic literature review of RBAC extension models that began with 1,716 papers of which 27 were deemed as primary sources for inclusion.
\\
\\
\noindent
{\bf Results}: We identified and classified the RBAC extension models into eight extension categories: Constraint, Context, Organization, Privacy, Task, Spatio-Temporal, Spatial, and Temporal. Only 8 of the 27 papers provided an implementation of their model in the form of an enterprise application or prototype. The primary domains that inspired extensions were the medical domain with 9 of the 27 models, enterprise workflows with 5 of 27, and mobile computing with 5 of 27.
\\
\\
\noindent
{\bf Conclusions}: Our literature review shows that all eight of the RBAC extension categories we identified deal with context whereby the privileges provided to a role are environmental depending upon factors such as location or time. 
RBAC extension model evaluation lacks a consistent set of metrics and evaluation of current models was found to range from providing scenario examples of the model in action to comparison to the RBAC standard.
The magnitude and scope of extensions to the RBAC standard suggests a revised standard may be beneficial but in the meantime this work can serve as a starting place for researchers and practitioners.

\end{abstract}
