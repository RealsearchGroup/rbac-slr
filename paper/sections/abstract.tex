\begin{abstract}

\noindent
{\bf Context}: Since the introduction of the role based access control (RBAC) standard in the late 1990's, 
RBAC has become an increasingly popular access control mechanism for applications such as 
web applications, mobile computing and databases. RBAC restricts access to resources 
based on the identity of subjects connected to logical groupings of permissions.  
The introduction of computing into new domains has led to proposals for extensions to the standard
RBAC model; for example, defining additional constraints among roles, adding context or defining new heirarchy relationships.
\\
\\
\noindent
{\bf Aim}: \textit{Researchers looking to expand on the body of knowledge surrounding extensions to RBAC need a starting place to prevent re-invention of the wheel.  
Developers need to find and select an access control mode based on their project requirements. 
The goal of this work is to provide practioners an assessment of the current extensions to RBAC, and researchers with insights into the state of RBAC extension models and how they are evaluated.}
\\
\\
\noindent
{\bf Method}: The authors conducted a systematic literature review by collecting and synthesizing relevant research
papers in the area of RBAC extensions. We initially collected 1716 papers from electronic digital libraries
IEEE, Google Scholar, CiteSeerX and ACM and selected 29 primary sources.  We performed a comparative 
analysis of the primary sources to find relationships among extended models and analyze the state of RBAC extension evaluations.
\\
\\
\noindent
{\bf Results}: Our results showed that extensions to RBAC fall into a number of categorizations that in turn all
fall under the single category of context based extensions.  We look into the motivations behind RBAC extensions and
the domains that are leading factors in developing these newer models. We also quantify the current evaluation methods
used when RBAC extension models are presented to the research community.
\\
\\
\noindent
{\bf Conclusions}: 

\end{abstract}
