\begin{abstract}

Since the introduction of the Role-based access control (RBAC) standard in the late 1990's, 
RBAC has become an increasingly popular access control mechanism for enterprises such as 
web applications, mobile computing and databases. RBAC restricts access to resources 
based on the identity of subjects connected to logical groupings of permissions called roles.  
The introduction of computing into new domains has led to proposals for extensions to the standard
RBAC model; for example, defining additional constraints among roles, adding context 
or defining new hierarchy relationships.  \textit{The goal of this work is to provide practioners an assessment of the state of extended models of RBAC and researchers with insights into the lack of robust evaluation of RBAC extension models.}
We conducted a systematic literature review by collecting and synthesizing relevant research
papers in the area of RBAC extensions. We initially collect XXXX papers from sources such as IEEE and ACM 
websites and selected X primary sources.  We performed a comparative analysis of the primary sources to 
find relationships among extended models and analyze the state of RBAC extension evaluations.

\end{abstract}
