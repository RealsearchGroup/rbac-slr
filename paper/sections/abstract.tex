\begin{abstract}

\noindent
{\bf Context}: Since the introduction of the role based access control (RBAC) standard in the late 1990's by the National Institute of Standards and Technology (NIST),
computing in new domains has led to proposals for extensions to the standard RBAC model; for example, defining additional constraints among roles, adding context or defining new hierarchy relationships.
\\
\\
\noindent
{\bf Aim}: The goal of our work is to aid practitioners and researchers in choosing an extension to the RBAC standard model, and in understanding
how extensions to RBAC are evaluated by providing an assessment of the state of extensions to the RBAC standard model through: establishment of a 
set of extension categories, an examination of the state of the art in evaluations of extension models and a breakdown of the reasoning and motivations
that have led to extension of the RBAC standard.
\\
\\
\noindent
{\bf Method}: We performed a systematic literature review that yielded 1716 papers, of which 28 were deemed primary sources for inclusion as model extensions to the RBAC standard model. 
\\
\\
\noindent
{\bf Results}: Our results show that extensions to RBAC can be classified under eight categories: Constraint, Context, 
Organization, Privacy, Task, Spatio-Temporal, Spatial and Temporal. We found that 12 models presented no evidence
of an evaluation, eight models used example scenarios and 6 others provided complexity and time analysis.  We identified only
eight of the 28 papers provided an implementation of their model in the form of an enterprise application or prototype. We found that the primary domains 
that inspired extensions were the medical domain with nine of the 28 models, enterprise workflows with five of 28, and mobile computing with five of 28.
\\
\\
\noindent
{\bf Conclusions}: The extensions to RBAC found by our systematic literature review fall into eight categories that upon closer inspection can be 
aligned under the single category of context. Our literature review showed that the state of evaluation of RBAC model evaluation is limited with one
paper doing an evaluation comparing itself to the core RBAC model, eight presenting example 
scenarios of their model in action, and 12 of the 28 models lacking any identifiable evaluation.

\end{abstract}
