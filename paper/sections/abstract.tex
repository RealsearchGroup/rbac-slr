\begin{abstract}

\noindent
{\bf Context}: Since the National Institute of Standards and Technology (NIST) proposed Role-Based Access Control (RBAC) standard using RBAC Reference Model in the late 1990's, computing in new domains has led to proposals for extensions to the RBAC Reference Model; for example adding context around permission granting and role activation.
\\
\\
\noindent
{\bf Aim}: The goal of our work is to aid practitioners and researchers in choosing an RBAC extension model, and in understanding
how RBAC extension models are evaluated by providing an assessment of the state of the RBAC extension models. We accomplished this through
establishment of a set of extension categories, an examination of the state of the art in evaluations of the RBAC extension models, and a breakdown of the motivations that have led to the RBAC extension models.
\\
\\
\noindent
{\bf Method}: We performed a systematic literature review that yielded 1,716 papers, of which 28 were deemed primary sources for inclusion as the RBAC extension models. 
\\
\\
\noindent
{\bf Results}: Our results show that the RBAC extension models can be classified under eight categories: Constraint, Context, 
Organization, Privacy, Task, Spatio-Temporal, Spatial, and Temporal. We identified only
8 of the 28 papers provided an implementation of their model in the form of an enterprise application or prototype. We found that the primary domains 
that inspired extensions were the medical domain with 9 of the 28 models, enterprise workflows with 5 of 28, and mobile computing with 5 of 28.
\\
\\
\noindent
{\bf Conclusions}: When examined as an entire group, the eight categories can be aligned under the single category of context. 
Our literature review shows that the state of evaluation of RBAC model evaluation is limited with 1
paper doing an evaluation comparing itself to the RBAC Reference Model, 8 presenting example 
scenarios of their model in action, and 12 of the 28 models lacking an evaluation that we could discern. The landscape of the RBAC extension models serves as a basis from which a new, and vetted, RBAC extension model should be derived.

\end{abstract}
