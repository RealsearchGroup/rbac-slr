\begin{abstract}

\noindent
{\bf Context}: Since the introduction of the role based access control (RBAC) standard in the late 1990's, 
RBAC has become an increasingly popular access control mechanism for applications such as 
web applications, mobile computing and databases. RBAC restricts access to resources 
based on the identity of subjects connected to logical groupings of permissions.  
The introduction of computing into new domains has led to proposals for extensions to the standard
RBAC model; for example, defining additional constraints among roles, adding context or defining new heirarchy relationships.
\\
\\
\noindent
{\bf Aim}: Our work aims to aid researchers looking to expand on the body of knowledge surrounding extensions to RBAC need a starting place to prevent re-invention of the wheel and
developers needing to find and select an access control mode based on their project requirements. 
\textit{The goal of this work is to provide practioners an assessment of the current extensions to RBAC to aid with choosing a model that meets their requirements, and researchers with insights into the state of RBAC extension models and how they are evaluated.}
\\
\\
\noindent
{\bf Method}: We conducted a systematic literature review by collecting and synthesizing relevant research
papers in the area of RBAC extensions. The review we performed yielded 1716 papers, of which 29 were deemed primary sources 
for inclusion as model extensions to the RBAC standard model. The papers we collected were from electronic digital libraries
IEEE, Google Scholar, CiteSeerX and ACM based on a set of exclusion and inclusion criteria.  
We performed a comparative analysis of the primary sources to find relationships among extended models and to analyze the state of RBAC extension evaluations.
\\
\\
\noindent
{\bf Results}: Our results showed that extensions to RBAC fall into a number of categories that in turn all
fall under the single category of context based extensions.  We look into the motivations behind RBAC extensions and
the domains that are leading factors in developing these newer models. We also quantify the current evaluation methods
used when RBAC extension models are presented to the research community.
\\
\\
\noindent
{\bf Conclusions}: The extensions to RBAC fall into a number of categorizations with Organization, Privacy, Resource, Task, Spatio-Temporal, Spatial and Temporal falling under the general category of context.
Domains, such as healthcare and mobile computing, were idenitfied as motivations behind the development of extensions to the RBAC model.  Our literature review showed that the state of RBAC model evaluation could benefit from the research community given most model evaluations seen within the papers were based on hypothetical situations with little to no case studies or implementations in practice.

\end{abstract}
