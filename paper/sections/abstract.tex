\begin{abstract}

\noindent
{\bf Context}: Since the US National Institute of Standards and Technology (NIST) proposed Role-Based Access Control (RBAC) as documented in the RBAC Reference Model in the late 1990's, domain-specific extensions have been proposed.  For example, the mobile domain has surfaced a need for permission granting and role activation.
\\
\\
\noindent
{\bf Aim}: The goal of our work is to aid practitioners and researchers in choosing an RBAC extension model, and in understanding
how RBAC extension models are evaluated by providing an assessment of the state of the RBAC extension models.
\\
\\
\noindent
{\bf Method}: We performed a systematic literature review that began with 1,716 paper of which 28 were deemed primary sources for inclusion as the RBAC extension models. We established a set of extension categories, examined the state of the art in evaluations of the RBAC extension models, and categorized the motivations that have led to the RBAC extension models.
\\
\\
\noindent
{\bf Results}: We classified the RBAC extension models into eight extension categories: Constraint, Context, 
Organization, Privacy, Task, Spatio-Temporal, Spatial, and Temporal. only 8 of the 28 papers provided an implementation of their model in the form of an enterprise application or prototype. The primary domains 
that inspired extensions were the medical domain with 9 of the 28 models, enterprise workflows with 5 of 28, and mobile computing with 5 of 28.
\\
\\
\noindent
{\bf Conclusions}: All of the eight RBAC extension categories deal with context. 
Our literature review shows that the state of evaluation of RBAC model evaluation is limited with 1
paper doing an evaluation comparing itself to the RBAC Reference Model, 8 presenting example 
scenarios of their model in action, and 12 of the 28 models lacking an evaluation that we could discern. The magnitude of extensions to the RBAC standard suggests a revised standard may be beneficial.

\end{abstract}
