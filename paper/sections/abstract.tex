\begin{abstract}

\noindent
{\bf Context}: Since the introduction of the role based access control (RBAC) standard in the late 1990's by the National Institute of Standards and Technology (NIST),
RBAC has become an increasingly popular access control mechanism for applications such as 
web applications, mobile computing and databases. RBAC restricts access to resources 
based on the identity of subjects connected to logical groupings of permissions.  
The introduction of computing into new domains has led to proposals for extensions to the standard
RBAC model; for example, defining additional constraints among roles, adding context or defining new heirarchy relationships.
\\
\\
\noindent
{\bf Aim}: The goal of our work is to aid practioners and researchers in choosing an exentsion to the RBAC standard model, and in understanding
how extensions to RBAC are evaluated by providing an assessment of the state of extensions to the RBAC standard model through: establisment of a 
set of extension categories, an examination of the state of the art in evaluations of extension models and a breakdown of the reasoning and motivations
that have led to extension of the RBAC standard.
\\
\\
\noindent
{\bf Method}: We conducted a systematic literature review by collecting and synthesizing relevant research
papers in the area of RBAC extensions. The review we performed yielded 1716 papers, of which 29 were deemed primary sources 
for inclusion as model extensions to the RBAC standard model. The papers we collected were from electronic digital libraries
IEEE, Google Scholar, CiteSeerX and ACM based on a set of exclusion and inclusion criteria.  
We performed a comparative analysis of the primary sources to find relationships among extended models and to analyze the state of RBAC extension evaluations.
\\
\\
\noindent
{\bf Results}: Our results showed that extensions to RBAC can be classified under seven categories: Constraint, Context, 
Organization, Privacy, Resource, Task, Spatio-Temporal, Spatial and Temporal. We found a range of motivations behind RBAC extensions from 
lack of fine-grained controls to a need for context placed on permissions, roles and assignment.  We found that 12 models presented no evidence
of an evaluation, 8 models used example scenarios and 6 others provided some complexity and time analysis.  Of the 29 papers, we identified only
8 that provided in an implementation of their model in the form of an enterprise application or prototype. We found that the primary domains 
that inspired were the model domain with 9 of the 29 models, enterprise workflows with 5 of 29, and mobile computing with 5 of 29.
\\
\\
\noindent
{\bf Conclusions}: The extensions to RBAC found by our systematic literature review fall into seven categoriies that upon closer inspection can be 
aligned under the single category of context. Two domains, healthcare and mobile computing, were idenitfied as predominant motivations behind the 
development of extensions to the RBAC model. Our literature review showed that the state of evaluation of RBAC model evaluation is limited with 1 
paper doing an evaluation comparing itself to the core RBAC model, 12 of the 29 models lacking any identifiable evaluation and 8 presenting example 
scenarios of their model in action, the predominant evaluation method.

\end{abstract}
