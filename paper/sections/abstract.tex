\begin{abstract}

\noindent
{\bf Context}: Since the National Institute of Standards and Technology (NIST) proposed Role-Based Access Control (RBAC) standard using RBAC reference models (RBAC$_{m}$ in short) in the late 1990's, computing in new domains has led to proposals for extensions to the RBAC$_{m}$; for example adding context around permission granting and role activation.
\\
\\
\noindent
{\bf Aim}: The goal of our work is to aid practitioners and researchers in choosing an extension to the RBAC$_{m}$, and in understanding
how extensions to RBAC$_{m}$ are evaluated by providing an assessment of the state of extensions to the RBAC$_{m}$. We accomplished this through: 
establishment of a set of extension categories, an examination of the state of the art in evaluations of extension models, and a breakdown of the 
motivations that have led to extension of the RBAC$_{m}$.
\\
\\
\noindent
{\bf Method}: We performed a systematic literature review that yielded 1,716 papers, of which 28 were deemed primary sources for inclusion as extensions to the RBAC$_{m}$. 
\\
\\
\noindent
{\bf Results}: Our results show that extensions to the RBAC$_{m}$ can be classified under eight categories: Constraint, Context, 
Organization, Privacy, Task, Spatio-Temporal, Spatial, and Temporal. We identified only
eight of the 28 papers provided an implementation of their model in the form of an enterprise application or prototype. We found that the primary domains 
that inspired extensions were the medical domain with nine of the 28 models, enterprise workflows with five of 28, and mobile computing with five of 28.
\\
\\
\noindent
{\bf Conclusions}: When examined as an entire group, the eight categories can be aligned under the single category of context. 
Our literature review shows that the state of evaluation of RBAC model evaluation is limited with one
paper doing an evaluation comparing itself to the RBAC$_{m}$, eight presenting example 
scenarios of their model in action, and 12 of the 28 models lacking an evaluation that we could discern. The landscape of extensions
to the RBAC$_{m}$ serves as a basis from which a new, and vetted, extended RBAC reference models should be derived.

\end{abstract}
