\section{Results} \label{sec:results}

\subsection{Search}

For our systematic literature review, the search strategy was applied to four search engines and three search strings based on popularity and database size.  This produced 12 different data sets that are combining for overlapping papers resulted in 1716 candidates.  The paper count for each search engine and criteria are shown in Table \ref{tab:search_results}.

\begin{table}
\centering
\begin{tabular}{|p{3.5cm}|p{1.25cm}|p{4.25cm}|p{4.25cm}|p{1cm}|}

\hline
\textbf{Search Engines} & 
\textbf{RBAC} & 
\textbf{role based access control} & 
\textbf{role-based access control} & 
\textbf{Total}
\\\hline

Google Scholar & 651 & 213 & 435 & 1299 \\\hline
ACM Portal & 500 & 20 & 720 & 1240 \\\hline
IEEExplore & 200 & 40 & 230 & 470 \\\hline
CiteSeerX & 100 & 100 & 150 & 350 \\\hline
 &  &  &  & \\\hline
Totals & 1451 & 373 & 1535 & 3359 \\\hline
Combined &  &  &  & \textbf{1716} \\\hline

\end{tabular}
\caption{Paper counts after applying search strategy}
\label{tab:search_results}
\end{table}

TODO: Add metrics for each portion of the elimination

\subsection{Extraction}

After selection of primary sources, the next step was to extract data from each paper that pertained to our research questions in order to look for trends.  The first step was to take the individual data generated from the final elimination round and organize this information around the research questions.  During the paper reading round, and resulting data, the fact that the papers were logically falling into a number of categorizations became evident.  Thus, the first step undertaken was to answer the question of what categories exist for the RBAC model extensions and what papers feel into what categories.  Each paper deemed a primary source and the papers associated categorization are shown in Table \ref{tab:categorization}.

\begin{table}
\centering
\begin{tabular}{|p{12.5cm}|p{3cm}|}
\hline
\textbf{Paper} & \textbf{Category}
\\\hline
Alam, M. and Hafner, M. and Breu, R. 2006 \cite{alam06:constraint} & Constraint \\\hline
Tzelepi, Sofia K. and Koukopoulos, Dimitrios K. and Pangalos, George 2001 \cite{tzelepi01:flexible} & Context \\\hline
Haibo, SHEN and Fan, HONG 2005 \cite{haibo05:context} & Context \\\hline
Damian G. Cholewka and Reinhardt A. Botha and Jan H. P. Eloff 2000 \cite{cholewka00:acontext-sensitive} & Context \\\hline
Huang, X. and Wang, H. and Chen, Z. and Lin, J. 2006 \cite{huang06:pervasive} & Context \\\hline
Motta, G.H.M.B. and Furuie, S.S. 2003 \cite{motta03:contextual} & Context \\\hline
Bao, Y. and Song, J. and Wang, D. and Shen, D. and Yu, G. 2008 \cite{bao08:role} & Context \\\hline
Yamazaki, W. and Hiraishi, H. and Mizoguchi, F. 2004 \cite{yamazaki104:designing} & Context \\\hline
Jian-min, H. and Xi-yu, L. and Hui-qun, Y. and Jun, T. 2008 \cite{han08:extended} & Context \\\hline
Thein, N. and others 2011 \cite{thein2011leveraging} & Context \\\hline
Zou, D. and He, L. and Jin, H. and Chen, X. 2009 \cite{zou2009crbac} & Context \\\hline
Hasebe, K. and Mabuchi, M. and Matsushita, A. 2010 \cite{hasebe10:capability} & Delegation \\\hline
Zhang, Z. and Zhang, X. and Sandhu, R. 2006 \cite{zhang06:collaborative} & Organizational \\\hline
Ni, Q. and Trombetta, A. and Bertino, E. and Lobo, J. 2007 \cite{ni2010privacy} & Privacy \\\hline
Masoumzadeh, A. and Joshi, J. 2008 \cite{masoumzadeh2008purbac} & Privacy \\\hline
Zhao, Y. and Zhao, Y. and Lu, H. 2008 \cite{zhao2008flexible} & Resource \\\hline
Bertino, E. and Catania, B. and Damiani, M.L. and Perlasca, P. 2005 \cite{damian2007geo} & Spatial \\\hline
Ray, I. and Kumar, M. and Yu, L. 2006 \cite{ray07:spatio} & Spatial \\\hline
Hansen, F. and Oleshchuk, V. 2003 \cite{hansen2003spatial} & Spatial \\\hline
Aich, S. and Sural, S. and Majumdar, A. 2007 \cite{aich07:STARBAC} & Spatio-Temporal \\\hline
Chen, L. and Crampton, J. 2008 \cite{chen08:spatio-temporal} & Spatio-Temporal \\\hline
Samuel, A. and Ghafoor, A. and Bertino, E. 2007 \cite{samuel07:spatio-temporal} & Spatio-Temporal \\\hline
Chandran, S. and Joshi, J. 2005 \cite{chandran05:llt} & Spatio-Temporal \\\hline
Ray, I. and Toahchoodee, M. 2007 \cite{ray07:spatio} & Spatio-Temporal \\\hline
Aich, S. and Mondal, S. and Sural, S. and Majumdar, A. 2009 \cite{aich09:role} & Spatio-Temporal \\\hline
Yao, L. and Kong, X. and Xu, Z. 2008 \cite{yao2008task} & Task \\\hline
Zhang, S. and Chen, X. and Hou, G. 2009 \cite{zhou2007team} & Task \\\hline
Oh, S. and Park, S. 2003 \cite{oh2003task} & Task \\\hline
Joshi, J.B.D. and Bertino, E. and Latif, U. and Ghafoor, A. 2005 \cite{joshi05:generalized} & Temporal \\\hline

\end{tabular}
\caption{Primary sources grouped by categorization}
\label{tab:categorization}
\end{table}

Given that there were multiple papers for some categories, the researchers decided to tackle all further research questions by first analyzing the research question on a per category basis and then looking across all categories for generalization and trends.

\section{Category Definitions} \label{sec:terms}

We found that different definitions for the same terms. Therefore, we next describe
terms and definitions.

\begin{itemize}
%	\item Access Control Model:
%	\item Access Control Policy:
%	\item Access Control Requirements:
%	\item Rules:
%	\item Attributes-Based Access Control:

%http://www.hpi.uni-potsdam.de/fileadmin/hpi/Forschung/Publikationen/Dissertationen/Diss_Zhou.pdf
% A Task-Role Based Access Control Model with Multi-Constraints
	\item Task-role-based access control: a task is a fundamental unit of a business activity. Different from core RBAC, in task-role-based access control model, roles are not directly associated with permissions. Roles are first associated with tasks, which are associated permissions. For example, the employee role is associated with a task, which is to write a report. Then, this task is associated with a permission.

% Team and Task Based RBAC Access Control Model	
	\item Team-based access control: a team encapsulates a group of users, who have various roles. A team member can activate only roles that can be assigned to a team. Moreover, team-based access control may decide maximum number of permissions such as roles (that are assigned to a team) having a maximum number of permissions to the team.			
%	\item Purpose-Based Access Control:
%	\item Agent-based access control:						
\end{itemize}



\begin{itemize}

%	\item Role Hierarchies:

	\item Delegation: delegation is to assign permissions (to access specific resources) available to a user to another user.
	A employee $E_1$ may delegate her/his permissions to another employee $E_2$ while $E_1$ is on leave. 
	
%	 obligations are requirements, which should be fulfilled before or after authorization decision is enforced.
%	Consider that a user has permission to access specific resources. For example, obligation is that the user should complete
%	her/his office duty before accessing the resources.
	
	\item Obligations: obligations are requirements, which should be fulfilled before or after authorization decision is enforced.
	Consider that a user has permission to access specific resources. For example, obligation is that the user should complete
	her/his office duty before accessing the resources.
	
	\item Inheritance:
	Inheritance defines an inheritance relationship among attributes such as roles. For example, the role structure for a company use
	employee role for employees. Department manager may inherit all permissions of the employee role. (Role-Based Access Control by F. Ferrailolo et al.)
	
	\item Static Separation of Duty (SSoD): SSoD restricts the conflicting-role assignments statically that are associated with a user. On situations
	where multiple roles can be associated with a single user and roles $Role_A$ and $Role_B$ are conflicting each other, no permission is given to a user who is assigned to both $Role_A$ and $Role_B$. SSoD is known to be too rigid for practical use in cases where a user should have permissions as either $Role_A$ and $Role_B$.

	\item Dynamic Separation of Duty: Dynamic SoD (DSD) is known to be
more flexible than SSD. DSD restricts the conflicting-role assignments dynamically that are associated with a user. On situations
	where multiple roles can be associated with a single user and , given a context, roles $Role_A$ and $Role_B$ are conflicting each other dynamically, no permission is given to a user.
	\end{itemize} 

We define temporal and spatial constraints as follows:

\begin{itemize}
	\item Temporal Constraints: Temporal constraints are time-based constraints in specifying access
	control policies. For example, in organizations, 	periodic temporal durations are enforced while a
	specific role is permitted to conduct an action. Consider that part-time employee works only from 9:00 a.m. to 3:00 p.m.
	In such cases, the part-time employee role should access required resources during the interval. 
	Temporal constraints can incorporate either on roles, user-role assignments, or role-permission assignments.   
	
	 
	\item Spatial Constraints: Spatial constraints are location-based constraints in specifying access
	control policies. For example, in organizations, 	locations are enforced while a
	specific role is permitted to conduct an action. Consider that part-time employee works only in specific location.
	In such cases, the part-time employee role should access required resources only when the user is in the location. 
	Spatial constraints can incorporate either on roles, user-role assignments, or role-permission assignments. 
	
\end{itemize}
