\section{Discussion} \label{sec:discussion}
%\subsection{Find appropriate extended Role-based access control models for a system}

This section discusses two issues: first one is about evaluation of extended role-based access control models and second one is to implement security system based on access control models.

\subsection{Metrics for Extended Access Control Models Evaluation }

Researchers provide metrics for access control system evaluation or security system in general~\cite{hu2012:NIST7874}. They collected metrics, which can be applicable for not only access control models, but also requirements and system implementation, and extended application such as verification and testing. From the metrics, we select metrics that can be applicable for evaluation of extended RBAC models as follows:

\begin{itemize}
	\item \textbf{Ease of Privilege Assignments}: this metric measures number of steps for assigning, changing, deleting a privilege for users and roles when an end-user or administrators manage RBAC. If one takes more steps for such a task, it may give more chance to make human or system errors due more steps. If you takes less step, it may reduce to make mistakes.  While extended RBAC may has similar features since from RBAC, we measure worst cases after extended, and gap between these two models.
	
	\item \textbf{Syntactic and semantic support for specifying AC rules}: this metric checks whether privileges and constraints can be specified using complex logic expressions such as AND, OR, $\leq$ when specifying constraints. This support helps specify complex and flexible access controls. For example, given a spatial RBAC model, one may measure whether spatial constraints can be specified using various logic expressions.

	\item \textbf{Delegation of administrative capabilities}: this metric measure how easy and secure when an access control model supports for privilege delegation of administrators to another user.

	\item \textbf{Capable of combining different policy models}: this metric measures whether an access model has a feature to combine more than two policies.

	\item \textbf{Capable of conflict resolution}: These metric checks whether an access model has a feature to resolve conflicts when two rules in a policy may be applicable for the same request, but evaluate different decisions: One grants permission and the other one denies permission.

	\item \textbf{Least privilege principle support}: This metric checks whether an access model supports for least privilege principle where a subject can access least set of information and resources that are necessary for its legitimate functioning. Least privilege principle support can reduce damage when malicious users gain legitimate privileges.

	\item \textbf{Separation of Duty (SoD)}: This metric checks whether an access model supports for static SoD and/or dynamic SoD. Note that NIST Standard supports both static SoD and dynamic SoD described in Section~\ref{sec:core-rbac}.
\end{itemize}

One can measure these metrics for a given extended access control model. These metrics focus on comparing specific features support of access control models.


\subsection{Extended RBAC Adoption}

In order to adopt a model in practice, software developers should implement the model in real system environments based on intended use of the models. Typically, a model provides abstract formal representation of extended RBAC and its operation. To configure intended use (i.e., which subjects can access which resources), we define an access control policy, which specifies high-level rules such as which subjects can access to which resources. Given an access control model and policy, security mechanism provides low-level implementation in a system, which controls access of roles.

A formal representation of access control models is a critical step of designing high-level abstraction where which elements are used and how elements are operated.
One of the objectives of a formal representation is to help ensure the correct behaviors through formal verification. Given the formal representation of access control models, software developers may prove of properties (e.g., safety, consistency and completeness) and check whether these properties are satisfied. 

While the concept of the RBAC model is clear and can be applicable to any system, it is challenging to implement the security mechanism of RBAC because the RBAC model is can be interpreted in more than one interpretation due to its complexity. In order to bridge gap between RBAC standard and its implementation, NIST published RBAC Implementation and Interoperability standard (RIIS/ ANSI INCITS-459) in 2011. This standard specifies how to implement RBAC system, which is consistent with NIST RBAC standard. Moreover, this standard describes interoperability specification where one RBAC implementation can be translated to another one.

This standard does not provide a specific guideline for implementation of various extended RBAC models.
Moreover, when software developers implemented extended RBAC models, they require clear security requirements of extended role-based access control, which identifies security objectives, intended environments, and assumed threats. Moreover, software developers understand how this extended RBAC overlaps with other access control models when more than two access control models are integrated into a single system. A survey~\cite{o20102010} shows that over 50\% of users at organizations with more than 500 employees are given some of their permissions to access resources based on RBAC. To adopt RBAC into a system, organizations often use hybrid approaches, which combine RBAC and access control lists since specific user types, systems, and workflow may not be effective to manage access based on roles. Therefore, when RBAC is extended and combined with other access control models, software developers often require to provide a formal representation, which can support of proof of the model. Then, software developers implement extended models based on NIST standard of RBAC implementation. Especially, software developers should understand intended use of extended access control models to meet a new system requirement. For example, for a spatial RBAC model, software developers incorporate additional spatial constraints, which can be either static or dynamic, in practice into the RBAC model.


