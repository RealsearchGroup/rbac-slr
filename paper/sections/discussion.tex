\section{Discussion} \label{sec:discussion}

The research questions we identified, presented results for and analyzed provide a view of the RBAC extension model landscape. 
By looking across all the research questions we can arrive at cross-cutting concerns and identify areas that may benefit from future work.
We discuss motivations behind the RBAC extension models and guidelines for choosing an RBAC extension model.

\subsection{Motivations behind the RBAC extension models}

While the RBAC Reference Model is considered fundamental in any RBAC systems, the RBAC Reference Model has limitations on providing features such as dealing with context for emerging applications such as healthcare and mobile devices. For example, in cases where the RBAC extension model does not support the notion of specific constraints, the RBAC extension model incorporates additional entities with regards to context/constraints and their relations with existing entities of the RBAC Reference Model. The RBAC extension model provides general agreement on the definition for specific features. The RBAC extension model helps simplify theoretical modeling and practical implementation of features in the model. 

During the paper reading phase, we identify motivations of the RBAC extension models by category. We identify motivations by looking for issues regarding why the RBAC reference model was inadequate and how the authors addressed the issues. We describe motivations behind the RBAC extension models by category as follows:

\begin{itemize}

  \item \textbf{Context}: The RBAC Reference Model does not support the notion of context constraints related to changes in environments. Therefore, RBAC belongs to the static access control model, which may not capture changes in environments.
 
  \item \textbf{Constraint}: The RBAC Reference Model has limitations on its features such as delegation and role hierarchy. For example, partial inheritance in role hierarchy needs to be developed.
  
  \item \textbf{Organizational}: The RBAC extension model does not handle RBAC administrative tasks efficiently across multiple organizations. The model needs to reduce the administrative complexity of RBAC across multiple organizations.
   
  \item \textbf{Privacy}: The RBAC extension model does not support the notion of privacy. For example, the model lacks components, constraints, and obligations to handle privacy in RBAC.
  
  \item \textbf{Task}: The RBAC extension model does not support the notion of a task, team, purpose, and organizational roles, which help specify a business activity in enterprises.
    
  \item \textbf{Spatio-Temporal}: The RBAC extension model does not support spatial (location-based) and temporal (time-based) constraints, which specify role-assignment, role-activation and permissions based on location and time.
  
  \item \textbf{Spatial}: The RBAC extension model does not support the notion of spatial (location-based)  constraints, which specify role-assignment and permissions based on location.
        
  \item \textbf{Temporal}: The RBAC extension model does not support the notion of temporal (time-based)  constraints, which specify role-assignments and permissions based on time.
      
\end{itemize}
 
We found that motivations for RBAC extension models vary across category. In general, motivations are rooted in adding new entities or relationships to allow authorization flexibility in meeting the demands of emerging requirements.

\subsection{Adoption of RBAC Extension Models}

To adopt a model in practice, software developers would implement the model in real system environments based on intended use of the models. 
As shown by RQ5, all primary sources were found to provide an abstract formal representations of their extension model and the model's operation.
The presence of an abstract formal representation of the model stands as a starting place for developers to configure intended use (i.e. which subjects can access which resources), and define an access control policy (i.e. which specifies high-level rules such as which subjects can access to which resources). 

A formal representation of access control models is a critical step of designing high-level abstraction where which entities are used and how these entities are operated.
One of the objectives of a formal representation is to help ensure the correct behaviors through formal verification. 
Given the formal representation of access control models, software developers may prove of properties (e.g., safety, consistency and completeness) and check whether these properties are satisfied. 

While the concept of the Reference Model is clear and can be applicable to any system, it is challenging to implement the security mechanism of RBAC because the Reference Model is can be interpreted in more than one interpretation due to its complexity. 
To bridge gap between RBAC standard and its implementation, NIST published RBAC Implementation and Interoperability standard (RIIS/ ANSI INCITS-459) in 2011. 
This standard specifies how to implement RBAC system, which is consistent with NIST RBAC standard. 
Moreover, this standard describes interoperability specification where one RBAC implementation can be translated to another one.

This standard does not provide a specific guideline for implementation of various RBAC extension models.
Moreover, when software developers implemented RBAC extension models, they require clear security requirements of extended role-based access control, which identifies security objectives, intended environments, and assumed threats. 
Moreover, software developers understand how this RBAC extension model overlaps with other access control models when more than two access control models are integrated into a single system. 
A survey~\cite{o20102010} shows that over 50\% of users at organizations with more than 500 employees are given some of their permissions to access resources based on RBAC. 
To adopt RBAC into a system, organizations often use hybrid approaches, which combine RBAC and access control lists because specific user types, systems, and workflow may not be effective to manage access based on roles. 
Therefore, when RBAC is extended and combined with other access control models, software developers can provide a formal representation which can serve as support for proof of the model. 
Then, software developers implement RBAC extension models based on NIST standard of RBAC implementation. 
Especially, software developers understand intended use of RBAC extension models to meet a new system requirement. 
For example, for a spatial RBAC model, software developers incorporate additional spatial constraints, which can be either static or dynamic, in practice into the RBAC model.


