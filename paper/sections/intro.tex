\section{Introduction} \label{sec:introduction}

Innovation and the use of role based access control (RBAC) into new domains has led to scenarios that, by some accounts, the standard RBAC model cannot handle \cite{kuhn2010adding}.  
For example, Ni, Q. et al \cite{ni2010privacy} state that RBAC is ``not designed to enforce privacy policies and barely 
meet privacy protection requirements`` with the introduction of privacy concerns in domains such as healthcare. 
Since the introduciton of the NIST RBAC standard in 2000, extension models to the RBAC standard have appeared in the literature, each adding one or more features onto the core.  
A cursory examination of the extension model papers does not provide good comparisons. Further, no generalizations among models have been made which inherently defies the reasoning behind
creating a standard.

The RBAC standard was first introduced in the 1990s when the National Institute for Standards and Technology (NIST) 
requested that a unified standard be created by combining the Ferraiolo and Kuhn model \cite{ferraiolokuhn} with the framework 
proposed by Sandhu, et al \cite{sandhu1996role}.  In 2004, this standard was adopted as ANSI/INCITS 359-2004.  The development of a standard was inspired by an economic impact study done 
during the 1990s and later confirmed in 2011 \footnote{\url{http://csrc.nist.gov/groups/SNS/rbac/documents/20101219_RBAC2_Final_Report.pdf}}
that showed the cost savings of RBAC in implementation and maintenance.
RBAC is a generalized access control approach used for various applications including web services, 
database applications, and healthcare applications.  RBAC has advantages in maintaining and managing organization's security policies.  
For example, if a user is to access manager role's resources within a given organization, security policy administrators simply add the user to be associated with the manager role.
In an RBAC model, roles represent a group of users who are involved in a specific job function in an organization. RBAC assigns permissions 
of specific actions on resources to roles instead of individual users. Therefore, in order to gain a roles' permission on specific resources, 
users acquire appropriate roles first.  

\textit{The goal of this work is to provide practioners an assessment of the current extensions to RBAC to aid with choosing a model that meets their requirements, and researchers with insights into the state of RBAC extension models and how they are evaluated.} With respect to our goal, the authors addressed the following research questions:

\begin{itemize}
\setlength{\itemsep}{0.25pt}
\item What categorizations exist within extensions to RBAC?
\item What are the motivations behind extensions to RBAC?
\item Do the extensions to RBAC have corresponding implementations?
\item How are extensions to RBAC evaluated theoritically and in practice?
\item What domains have extensions to RBAC been created for?
\item What commonalities or generalizations exist across all categorizations?
\end{itemize}

The authors performed a systematic literature review designed to explore the current body of research in the area of extensions to the core RBAC model.  The review we performed yielded 1716, of which 29 were deemed primary sources for inclusion as model extensions to the RBAC standard model.  This review is intended to serve as a starting place for researcher's looking to tackle new problems in the realm of authorization and prevent re-invention of the wheel. For developer's looking to find a model to fit their seemingly unique access control needs, this review will provide a basis for comparison and easy look-up for what extension based model to use.  Further, the review provides insight into the state of evaluation, or lack thereof, within the RBAC extension model community.

Our research provides contributions to the community in the following ways:

\begin{itemize}
\setlength{\itemsep}{0.25pt}
\item Summarizes current extended RBAC research work and its contributions.
\item Guides a direction for a standard of extended RBAC. Understanding the categorization and the motivation of the existing research results helps decide a standard of extended RBAC.
\item Identifies the research challenges in the areas of security policies and suggest a future extension of RBAC.
\item Identifies the deficiencies in RBAC extension evaluation.
\end{itemize}

The rest of the paper is organized as follows.  Section 2 covers background and the core RBAC model. Section 3 lays out the process used in conducting the review. Sections 4-9 present the analysis and discussion of the research questions. Section 10 contains the final conclusions.
