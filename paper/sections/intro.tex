\section{Introduction} \label{sec:introduction}

Why is the base model of RBAC extended by newer models?

P1: Note history of the RBAC model, statistics, and official status

P2: Mention extensions exist in response to developments and new domains over time

P3: Mention audience, why they should care, and why exploring extensions space is important in relation to the reason the standard was originally concieved

P4: Talk about what this work is going to do, goal, brief process

P5: List goals or contributions

P6: Outline paper and sections


Role based access control (RBAC) was first introduced in the 1990 when the National Instittute for Standards and Technology (NIST) requested that a unified standard be created.  proposed standard RBAC model \cite{ferraiolo}

Role-based access control (RBAC) models \cite{ferraiolo} became popular used to govern access to critical resources.  In an RBAC model, roles represent a group of users who are involved in a specific job function in an organization. RBAC assigns permissions of specific actions on resources to roles instead of individual users.  Therefore, in order to gain roles' permission on specific resources, users acquire appropriate roles first.

RBAC is a generalized access control approach used for various applications including web services, database applications, and healthcare applications.  RBAC has advantages in maintaining and managing organization's security policies.  For example, if a user is to access manager role's resources within a given organization, security policy administrators simply add the user to be associated with the manager role.

Standard RBAC model considers only role-user association and role hierarchy.
Since standard RBAC model has limitations such as specifying environmental constraints or context information
Researchers developed extended models of RBAC to overcome the limitations.
However, as researchers often develop their own specialized extended models of RBAC,
their research cannot be generalized or compared with other research work appropriately.
As a result, researchers could take time on reinventing the wheel.
But how do we, as a community, ensure that a metric is suitable and acceptable for its intended purpose?

The goal of this work is to synthesize available research results on extended models of RBAC. We analyze their extended features and claimed research contributions to find limitations of current RBAC models and what extent of extended features by
comparing with similar research work.
We conducted a systematic literature review (SLR) to evaluate and interpret all available research relevant to a particular research question or topic area of interest.

Our research give benefits to a community as follows:

\begin{itemize}
\item Our work summarizes current extended RBAC research work and its contributions. By synthesizing the current results, our work shows a roadmap of current extended RBAC research.
\item Our work guides a direction for a standard of extended RBAC. Understanding the categorization and the motivation of the existing research results helps decide a standard of extended RBAC.
\item Our work shows a criteria in comparison among research results.
\item Our work helps identify the research challenges in the ares of security policies and suggest a future extension of RBAC.
\end{itemize}
