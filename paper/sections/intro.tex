\section{Introduction} \label{sec:introduction}

Innovation and the use of role based access control (RBAC) in new domains has led to scenarios that, by some accounts, the standard RBAC model cannot handle \cite{kuhn2010adding}.  
For example, Ni, Q. et al \cite{ni2010privacy} state that RBAC is ``not designed to enforce privacy policies and barely 
meet privacy protection requirements`` with the introduction of privacy concerns in domains such as healthcare. 
Since the introduciton of the NIST RBAC standard in 2000, extension models to the RBAC standard have appeared in the literature, each adding one or more features on top of the core RBAC model. 
A cursory examination of the extension model papers does not provide good comparisons for developers and researchers to build upon. 
Further, these extension models are each building upon and adding features to a standard that was designed to reduce the economic impact experienced by enterprises and increase interoperability \cite{o20102010}.

The RBAC standard was first introduced in the 1990s when the National Institute for Standards and Technology (NIST) 
requested that a unified standard be created by combining the Ferraiolo and Kuhn model \cite{ferraiolokuhn} with the framework 
proposed by Sandhu, et al \cite{sandhu1996role}.  
In 2004, this standard was adopted as ANSI/INCITS 359-2004 approved by American National Standards Institute (ANSI) and the InterNational Committee for Information Technology Standards (INCITS). 
The development of a standard was inspired by an economic impact study done during the 1990s and later confirmed in 2011 \footnote{\url{http://csrc.nist.gov/groups/SNS/rbac/documents/20101219_RBAC2_Final_Report.pdf}} that showed the cost savings of RBAC implementation and maintenance.
NIST found that IT developers and users in government and industry need consistent and uniform definition of RBAC features. 
Prior to the development of RBAC standard, vendors proposed and implemented their own defined RBAC features without general agreement on the unified definition of RBAC features. 
As a result, IT developers and users have uncertainty, confusion and a lack of interoperability of RBAC features. 
Due to this uncertainty and confusion, not only do the development and maintenance cost of RBAC increase but also frequencies of security violations.
RBAC standard address this issue by describing definition and functional specification of RBAC features using a reference model.

RBAC is a generalized access control approach used for various applications including web services, database applications, and healthcare applications.  
RBAC has advantages in maintaining and managing an organization's security policies.  
For example, if a user requires access to resources associated with the manager role within a given organization, security policy administrators need only associate the user with the manager role.
In the standard RBAC model, roles represent a set of permissions needed to perform a particular job function within an organization.  
Any number of users who are involved in that specific job function within the organization can then be assigned to the role to inherit the required access instead of assigning indvidual permissions to individual users. 
As the number of permissions and users scales upward, the ability to logically group both users and permissions into smaller numbers of entities becomes paramount for managing access control as an organization grows and evolves.

\textit{The goal of our work is to aid practioners and researchers in choosing an exentsion to the RBAC standard model, and in understanding
how extensions to RBAC are evaluated by providing an assessment of the state of extensions to the RBAC standard model through: establisment of a 
set of extension categories, an examination of the state of the art in evaluations of extension models and a breakdown of the reasoning and motivations
that have led to extension of the RBAC standard.} With respect to our goal, the authors addressed the following research questions:

\begin{itemize}
\setlength{\itemsep}{0.25pt}
\item RQ1: What categorizations exist within extensions to RBAC?
\item RQ2: What are the motivations behind extensions to RBAC?
\item RQ3: Do the extensions to RBAC have corresponding implementations?
\item RQ4: How are extensions to RBAC evaluated theoritically and in practice?
\item RQ5: What domains have extensions to RBAC been created for?
\item RQ6: What commonalities or generalizations exist across all categorizations?
\end{itemize}

The authors performed a systematic literature review designed to explore the current body of research in the area of extensions to the core RBAC model.  The review we performed yielded 1716, of which 28 were deemed primary sources for inclusion as model extensions to the RBAC standard model.  This review is intended to serve as a starting place for researcher's looking to tackle new problems in the realm of authorization and prevent re-invention of the wheel. For developer's looking to find a model to fit their seemingly unique access control needs, this review will provide a basis for comparison and easy look-up for what extension based model to use.  Further, the review provides insight into the state of evaluation, or lack thereof, within the RBAC extension model community.

Our research provides contributions to the community in the following ways:

\begin{itemize}
\setlength{\itemsep}{0.25pt}
\item Summarizes current extended RBAC research work and its contributions.
\item Guides a direction for a standard of extended RBAC. Understanding the categorization and the motivation of the existing research results helps decide a standard of extended RBAC.
\item Identifies the research challenges in the areas of security policies and suggest a future extension of RBAC.
\item Identifies the deficiencies in RBAC extension evaluation.
\end{itemize}

The rest of the paper is organized as follows.  Section 2 covers background and the core RBAC model. Section 3 lays out the process used in conducting the review. Sections 4-9 present the analysis and discussion of the research questions. Section 10 contains the final conclusions.
