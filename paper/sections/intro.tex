\section{Introduction} \label{sec:introduction}

Role based access control (RBAC) was first introduced in the 1990s when the National Institute for Standards and Technology (NIST) requested that a unified standard be created by combining the Ferraiolo and Kuhn model \cite{ferraiolokuhn} with the framework proposed by Sandhu, et al \cite{sandhu1996role}.  The development of a standard was inspired by an economic impact study done during the 1990s and later confirmed in 2011. In an RBAC model, roles represent a group of users who are involved in a specific job function in an organization. RBAC assigns permissions of specific actions on resources to roles instead of individual users.  Therefore, in order to gain roles' permission on specific resources, users acquire appropriate roles first.  RBAC is a generalized access control approach used for various applications including web services, database applications, and healthcare applications.  RBAC has advantages in maintaining and managing organization's security policies.  For example, if a user is to access manager role's resources within a given organization, security policy administrators simply add the user to be associated with the manager role.

Since the introduction of the standardized RBAC model, innovation and the spread of software engineering into new domains has led to scenarios that, by some accounts, the standard RBAC model cannot handle.  For example, RBAC is ill-equipped to handle the additional entities that need to be taken into consideration during authorization with the introduction of privacy concerns in domains such as healthcare  Thus, a series of extensions to the standard RBAC model have appeared in the literature, each adding one or more features.  However, as researchers often develop their own specialized extended models of RBAC, their research cannot be generalized or compared with other research work appropriately.

\textit{The goal of this work is to provide practitioners an assessment of the state of extended models of RBAC and researchers with insights into the lack of robust evaluation of RBAC extension models.} With respect to our goal, the authors addressed the following research questions:

\begin{itemize}
\item What categorizations exist within extensions to RBAC?
\item What are the motivations behind extensions to RBAC?
\item Do the extensions to RBAC have corresponding implementations?
\item How are extensions to RBAC evaluated theoritically and in practice?
\item What domains have extensions to RBAC been created for?
\item What commonalities or generalizations exist across all categorizations?
\end{itemize}

A systematic literature review is designed to yield quantifiable results across a body of research in order to draw out similarities, trends and deficiencies in an area as a whole.  The review we performed yielded 1716, of which XX were deemed primary sources for inclusion as model extensions to the RBAC standard model.  This review is intended to serve as a starting place for researcher's looking to tackle new problems in the realm of authorization and prevent re-invention of the wheel. For developer's looking to find a model to fit their seemingly unique access control needs, this review will provide a basis for comparison and easy look-up for what extension based model to use.  Further, the review provides insight into the state of evaluation, or lack thereof, within the RBAC extension model community.

Our research provides contributions to the community in the following ways:

\begin{itemize}
\item Summarizes current extended RBAC research work and its contributions
\item Guides a direction for a standard of extended RBAC. Understanding the categorization and the motivation of the existing research results helps decide a standard of extended RBAC.
\item Our work shows a criteria in comparison among research results.
\item Identify the research challenges in the areas of security policies and suggest a future extension of RBAC
\item Identify the deficiencies in RBAC extension evaluation
\end{itemize}

The rest of the paper is organized as follows.  Section 3 covers background and the core RBAC model. Section 3 lays out the process used in conducting the review. Section 4 provides the results of the search. Section 5 tackles the research questions and analyzes the results.  Section 6 contains the final conclusions.

\begin{itemize}
\item TODO: Add some references to the prevalence of RBAC in industry
\item TODO: Add link to RBAC standard documents
\end{itemize}
