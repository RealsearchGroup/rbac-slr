\section{Core Role Based Access Control} \label{sec:core-rbac}

P1: Describe RBAC entities

P2: Note the 4 levels of RBAC

P3: Concisely describe level 1

P4: Concisely describe level 2

P5: COncisely describe level 3

P6: Concisely describe level 4


Since the basis for our review is extensions to the core model, we will describe the core model, associated entities and other terminology encountered across the space of our review.  The NIST RBAC model proposed by Ferraiolo et al. and later adopted as the official standard for RBAC by the International Committee for Information Technology Standards (INCITS) consists of four basic entities:

\begin{itemize}
\item a set of users \emph{Users}: A user can be a person or an agent.
\item  a set of roles \emph{Roles}: A role is a collection of permissions to perform a specific job function in an organization.
\item a set of permissions \emph{Permissions}: A permission refers to an access mode that can be exercised on an object in the system and a session relates a user to possibly many roles.
\item a set of sessions \emph{Sessions}: In each session, a user can be assigned to some of the roles, only when the corresponding role is enabled for activation for that time.		
\end{itemize}

In the RBAC, a user can exercise a permission only if the user are assigned to a role.
In addition to the four basic components, two functions are defined:
the user role assignment (UA) and the role
permission assignment (PA) functions.
UA models assignment of users to roles.
PA models assignment of permissions to roles.

%Each user incorporates a session    
%The user function maps each session to a
%single user, whereas the role function establishes a
%mapping between a session and a set of roles activated
%by the corresponding user in the session.


\section{Terms and Definitions} \label{sec:terms}

NOTE: Not sure about this section since some of this is derived from our results

We found that different definitions for the same terms. Therefore, we next describe
terms and definitions.

\begin{itemize}
%	\item Access Control Model:
%	\item Access Control Policy:
%	\item Access Control Requirements:
%	\item Rules:
%	\item Attributes-Based Access Control:
	\item Task-Based Access Control:
%	\item Purpose-Based Access Control:
	\item Agent-Based Access Control:						
\end{itemize}



\begin{itemize}

%	\item Role Hierarchies:

	\item Obligations: obligations are requirements, which should be fulfilled before or after authorization decision is enforced.
	Consider that a user has permission to access specific resources. For example, obligation is that the user should complete
	her/his office duty before accessing the resources.
	
	\item Inheritance:
	Inheritance defines an inheritance relationship among attributes such as roles. For example, the role structure for a company use
	employee role for employees. Department manager may inherit all permissions of the employee role. (Role-Based Access Control by F. Ferrailolo et al.)
	
	\item Static Separation of Duty (SSoD): SSoD restricts the conflicting-role assignments statically that are associated with a user. On situations
	where multiple roles can be associated with a single user and roles $Role_A$ and $Role_B$ are conflicting each other, no permission is given to a user who is assigned to both $Role_A$ and $Role_B$. SSoD is known to be too rigid for practical use in cases where a user should have permissions as either $Role_A$ and $Role_B$. (Role-Based Access Control by F. Ferrailolo et al.)

	\item Dynamic Separation of Duty: Dynamic SoD (DSD) is known to be
more flexible than SSD. DSD restricts the conflicting-role assignments dynamically that are associated with a user. On situations
	where multiple roles can be associated with a single user and , given a context, roles $Role_A$ and $Role_B$ are conflicting each other dynamically, no permission is given to a user. (Role-Based Access Control by F. Ferrailolo et al.)
	\end{itemize} 

We define temporal and spatial constraints as follows:

\begin{itemize}
	\item Temporal Constraints: Temporal constraints are time-based constraints in specifying access
	control policies. For example, in organizations, 	periodic temporal durations are enforced while a
	specific role is permitted to conduct an action. Consider that part-time employee works only from 9:00 a.m. to 3:00 p.m.
	In such cases, the part-time employee role should access required resources during the interval. 
	Temporal constraints can incorporate either on roles, user-role assignments, or role-permission assignments.   
	(Role-Based Access Control by F. Ferrailolo et al.)
	 
	\item Spatial Constraints: Spatial constraints are location-based constraints in specifying access
	control policies. For example, in organizations, 	locations are enforced while a
	specific role is permitted to conduct an action. Consider that part-time employee works only in specific location.
	In such cases, the part-time employee role should access required resources only when the user is in the location. 
	Spatial constraints can incorporate either on roles, user-role assignments, or role-permission assignments. 
	
\end{itemize}

