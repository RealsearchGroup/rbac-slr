\section{Analysis} \label{sec:analysis}

\subsection{Research Question 1}

What are the deficiencies in the current RBAC model? \\

\subsection{Research Question 2}

What are the motivations behind RBAC extensions? \\

\subsection{Research Question 3}

Do the extension models have corresponding implementations in practice? \\

When designing and proposing a model targeted at a feature that is rooted in practical
usage by real software systems, bringing the model to life is strong evidence that the
proposed model can work in practice.  The concept of authorization, and access control
is rooted in a business need.  Thus, any access control model needs to be feasible
in the real world not just on paper.  We analyzed the primary sources to see how many
proposed models actually had implmenetations associated with them.  And quantified the
type of implementation.  Whether the implementation was for a real system, for a prototype
and/or used in a production environment.

Of the 29 papers surveyed, there was a significant lack of implementation with 21 of the paper providing no
mention of a implementation or prototype.  Of the remianing 8 papers that did mention an implementation, half 
were simply prototypes developed by the authors while the other half were claimed to be implemented within a real
system.

\subsection{Research Question 4}

How are extended RBAC models evaluated in theory and in practice? \\

Providing evaluation of a proposed model is a key component in establishing the models validity.  The papers were examined
for evidence of evaluations ranging from performance to mathematical accuracy to application to real world scenarios.  Further,
for each proposed model, the reviewers looked for evidence that the authors made comparisons between their own model and the base
model as they pertained to claims made by the authors of why their model is needed.  The quantifiable evaluations looked for were:

\begin{itemize}
\item Time-based Performance
\item Complexity analysis
\item Comparison to standard RBAC
\item Mathematical accuracy
\item Example scenarios of the model in action
\item Experimental analysis of the model
\item Case study of the model in practice
\end{itemize}

Based on the diverse evaluation criteria, 12 models presented no evidence of an evaluation.  8 models presented example scenarios
and how application of their model would apply and resolve the situation.  6 of the models provided some form of performance
or complexity analysis of their model.  This included graphs of the model's time to determine authorization as the number of entities
grew, and the size of the role space for the extension model compared to standard RBAC. 4 models provided mathematical descriptions
and analsysis as a way to provide evaluation in the form of completeness.

The most widely used evaluation method was providing sample scenarios with accompanying workflows of how the extension model
would tackle those scenarios.  Much is left to the reader to assume of these types of evaluations, as the authors do not explicitly state
or show how the standard model is deficient in tackling said scenarios.

\subsection{Research Question 5}

What domains or scenarios serve as inspiration for these extensions? \\

Business needs have historically driven RBAC research and development.  The primary mode of evaluation for
model extensions has been the presentation of business scenarios in various domains and how the model
uniquely handles those particular scenarios.  Thus, looking for trends in the domains used in the example
scenarios might serve to illuminate a trend worth further examination into the reason for the explosion of
RBAC extensions.

Upon examination of the primary sources, the most prevalent domains were:

\begin{itemize}
\item What domains or scenarios serve as inspiration for these extensions?
\item Medical domain
\item Pervasive computing environments
\item Mobile devices
\item Large-scale organizations with many sub-departments
\item Enterprise, organization workflows
\end{itemize}

\subsection{Research Question 6}

What commonalities or generalizations exist across all categorizations? \\
