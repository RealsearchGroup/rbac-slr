\section{Motivations} \label{sec:motivations}

RQ2: What are the motivations behind RBAC extensions? \\

Core RBAC model provides an abstraction of authorization model based on user role assignment (UA) and the role
permission assignment (PA). Since this simple model may not provide a fine-grained access control for sophisticated
security mechanism, a variety of extended RBAC models have been proposed over the years to meet
security requirements. When administrators design a model, it is important to capture an important abstraction to help the model to be enforced in a system.

\begin{itemize}
\setlength{\itemsep}{0.25pt}
\item Core RBAC model needs additional contextual information and constraints to develop fine-grained policies in practice.
\item Core RBAC does not incorporate context. Therefore, RBAC belongs to static access control model, which may not capture changes in environments.
\item Core RBAC does not support for various constraints such as temporal and spatial constraints to design sophisticated policies on demand.
\item Core RBAC does not provide an abstraction to additional user-defined attributes	(e.g., task and team) and their association with existing attributes.
\item Core RBAC has limitation on delegation and role hierarchy. For example, partial inheritance in role hierarchy needs to be developed.  
\item Model needs to incorporate additional contextual information and constraints to develop fine-grained policies in practice.
\item Model needs to incorporate additional attributes such as
task, team, purpose, organizational roles and collaborative activities over existing attributes. Moreover, new association between attributes
are introduced. 
\item Model needs to improve existing features such delegation and role hierarchy to meet fine-grained access control. 
\end{itemize}

\subsection{Analysis and Discussion}


\section{Implementations} \label{sec:implementations}

RQ3: Do the extension models have corresponding implementations? \\

When designing and proposing a model targeted at a feature that is rooted in practical
usage by real software systems, bringing the model to life is strong evidence that the
proposed model can work in practice.  The concept of authorization, and access control
is rooted in a business need.  Thus, any access control model needs to be feasible
in the real world not just on paper.  We analyzed the primary sources to see how many
proposed models actually had implementations associated with them.  And quantified the
type of implementation.  Whether the implementation was for a real system, for a prototype
and/or used in a production environment.

\begin{table}
\centering
\begin{tabular}{ | c | c | }
\cline{1-2}

\textbf{Implementation Type} & \textbf{Paper Count} \\ \cline{1-2}
Enterprise Implementation & 4 \\ \cline{1-2}
Prototype Implementation & 4 \\ \cline{1-2}
No Implementation & 20 \\

\cline{1-2}
\end{tabular}
\caption{Implementation types found and the count of primary sources}
\label{tab:implementations}
\end{table}

Table \ref{tab:implementations} shows the breakdown of implementations found within the primary sources.
Of the 28 papers surveyed, there was a lack of implementation with 20 of the paper providing no
mention of a implementation or prototype.  Of the remaining 8 papers that did mention an implementation, half 
were simply prototypes developed by the authors while the other half were claimed to be implemented within a real
system.

\subsection{Analysis and Discussion}

The RBAC standard was designed with enterprises in mind such that when practioners implemented RBAC into their systems
there would be a reasonable assurance being based off a well thought out model.  As extensions to the standard model
come along, thought and time should be given to how features and nuances of their models may impact implementation
in order to achieve the same goals as the original standard.  The primary sources should a significant lack of implementation
with over 70\% of the models having no notion of attempting to implement them.  The bare minimum, as 4 papers did, should be
a prototype implementation of the model for review by both pracitioners and researchers. Of the models that produced an 
implementation within the enterprise world, two were from within the medical domain and two were implemented using web application technologies.  

\section{Evaluations} \label{sec:evaluations}

RQ4: How are extension to RBAC evaluated theoritically and in practice? \\

Providing evaluation of a proposed model is a key component in establishing the models validity.  The papers were examined
for evidence of evaluations ranging from performance to mathematical accuracy to application to real world scenarios.  Further,
for each proposed model, the reviewers looked for evidence that the authors made comparisons between their own model and the base
model as they pertained to claims made by the authors of why their model is needed.  The quantifiable evaluations looked for were:

\begin{itemize}
\setlength{\itemsep}{0.25pt}
\item Time-based Performance \cite{ni2010privacy}, \cite{aich09:role}
\item Complexity analysis \cite{bao08:role}, \cite{zhang06:collaborative}, \cite{chen08:spatio-temporal}, \cite{aich09:role}
\item Comparison to standard RBAC \cite{bao08:role}, \cite{zou2009crbac}, \cite{zhang06:collaborative}, \cite{zhao2008flexible}, \cite{ray07:spatio}
\item Mathematical modeling \cite{damian2007geo}, \cite{hansen2003spatial}, \cite{aich07:STARBAC}, \cite{chen08:spatio-temporal}, \cite{joshi05:generalized}
\item Example scenarios of the model in action \cite{alam06:constraint}, \cite{tzelepi01:flexible}, \cite{cholewka00:acontext-sensitive}, \cite{huang06:pervasive}, \cite{bao08:role}, \cite{jian2008extended}, \cite{yamazaki104:designing}, \cite{zou2009crbac}, \cite{ray07:spatio}, \cite{samuel07:spatio-temporal}, \cite{ray07:spatio}, \cite{joshi05:generalized}, \cite{yao2008task}, \cite{zhou2007team}, \cite{oh2003task}
\item Experimental analysis of the model
\item Case study of the model in practice \cite{motta03:contextual}
\end{itemize}

Based on the diverse evaluation criteria, 12 models presented no evidence of an evaluation.  8 models presented example scenarios
and how application of their model would apply and resolve the situation.  6 of the models provided some form of performance
or complexity analysis of their model.  This included graphs of the model's time to determine authorization as the number of entities
grew, and the size of the role space for the extension model compared to standard RBAC. 4 models provided mathematical descriptions
and analysis as a way to provide evaluation in the form of completeness.

The most widely used evaluation method was providing sample scenarios with accompanying workflows of how the extension model
would tackle those scenarios.  Much is left to the reader to assume of these types of evaluations, as the authors do not explicitly state
or show how the standard model is deficient in tackling said scenarios.

\subsection{Analysis and Discussion}


\section{Domains} \label{sec:domains}

RQ5: What domains have RBAC extensions been created for? \\

Business needs have historically driven RBAC research and development.  The primary mode of evaluation for
model extensions has been the presentation of business scenarios in various domains and how the model
uniquely handles those particular scenarios.  Thus, looking for trends in the domains used in the example
scenarios might serve to illuminate a trend worth further examination into the reasaon for the explosion of
RBAC extensions.  The authors identified domains presented within the primary sources by looking for example
scenarios cast within a particular domain or mention of domain requirements within the body of the paper.
We found that the domains mentioned and their associated sources are:

\begin{itemize}
\setlength{\itemsep}{0.25pt}
\item Medical domain \cite{alam06:constraint}, \cite{tzelepi01:flexible}, \cite{motta03:contextual}, \cite{ni2010privacy}, \cite{damiani2007geo}, \cite{hansen2003spatial}, \cite{samuel07:spatio-temporal}, \cite{aich09:role}, \cite{zhou2007team}
\item Pervasive computing environments \cite{huang06:pervasive}, \cite{chen08:spatio-temporal}, \cite{ray07:spatio}
\item Web applications \cite{masoumzadeh2008purbac}
\item Mobile computing \cite{thein2011leveraging}, \cite{zou2009crbac}, \cite{chandran05:llt}, \cite{ray07:spatio}, \cite{aich09:role}
\item Large-scale organizations with many sub-departments \cite{yamazaki04:designing}, \cite{han08:extended}, \cite{yao2008task}
\item Enterprise, organization workflows \cite{cholewka00:acontext-sensitive}, \cite{bao08:role}, \cite{zhang06:collaborative}, \cite{oh2003task}, \cite{joshi05:generalized}
\end{itemize}

The predominant domain for which extension models have been generated for is that of the medical domain with 9 of 28 mentioning scenarios or requirements of that industry.
Mobile computing and enterprise workflows were each represented by 5 papers claiming to be influenced by the requirements for access control within these domains. The final set
of domains was pervasive computing environments and large-scale organizations with 3 each and web applications with 1.  There were 4 papers without any direct mention of a domain
since Aich et al. \cite{aich09:role} fall under both the medical domain and mobile computing.

\subsection{Analysis and Discussion}

The medical domain produced the largest selection of papers when analyzing the domains influencing the proposals of extension models.  
Further, the authors noted that the categories associated with papers identifying the medical domain was not limited to one or two but cut across
each of the nine categories except for Organization and Resource. 
The cross-category nature of the medical domain papers appears indicative of the complex nature of medical applications and the requirements therein.
Given the growth of the research and development of medical applications over the past decade this result does not appear to be surprising. However,
the RBAC standard was originally created to reduce cost and increase interoperability - two goals of current regulation around the standardization
of electronic health record systems. The large number of proposed models, and the cross-category result stand in direct opposition of the goals
of both the RBAC standard and current regulations.

The RBAC standard has been re-enforced by the economic impact that standardization has had on enterprises needing to apply access control.  The
inclusion of extension models targeted at the enterprise workflow domain is indicative of the expansion of requirements for enterprises. Developers
and researchers should take care when looking at extension models designed to addresses the newer requirements of enterprise workflows in order to
achieve the same economic implementation and maintainability benefits the original standard model presents.

Mobile computing has seen a dramatic increase in the number of available devices, operating systems and applications since 1997 when the first smart phone
was introduced. 
The domain analysis results produced 5 papers that targeted extensions that are designed to address the requirements of mobile computing.
For a domain that was roots in personal and enterprise computing, protecting the data of both through access controls is paramount given their ubiquity.


\section{Generaliations} \label{sec:generalizations}

RQ6: What commonalities or generalizations exist across all categorizations? \\

Core or any extended role-based access control is used in various aspects of computer systems. In order to reduce efforts for modeling access control used in various applications, researchers often focus on developing generalized core concepts of access control.
We found that propositional logic is used to describe access control model across all categorizations. Propositional logic is concerned with propositions and their logical relationships. In propositional logic, simple (i.e., atomic) or compound condition at given context is evaluated to true or false based on specified rules and access control logic. Researchers are concerned to extend limited set of propositions specific to core RBAC to meet real-world scenarios such as dynamic constraints, temporal, or spatial constraints. However, semantic meanings of such propositions are various based on researchers' intention.

\subsection{Analysis and Discussion}
