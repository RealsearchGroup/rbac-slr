\section{Process} \label{sec:process}

The systematic literature review process was developed in whole prior to application and agreed upon by the researchers following recommendations from Kitchenham's suggested processes \cite{kitchenham2007guidelines}.  The systematic literature review process was broken down in to four stages and the rest of this section is broken down by each stage.  The stages were:

\begin{itemize}
\item Development of a search strategy
\item Elimination of papers based on title
\item Elimination of papers based on abstract
\item Elimination of papers based on content and matching to elimination criteria
\end{itemize}

\subsection{Search Strategy}

For the first phase of our systematic literature review, a search strategy for finding papers was developed.  The search strategy was executed by an automated comprehensive search taking as input a set of academic search engines and a list of search criteria.  The search performed was done in an automated way using a set of scripts to query and collect data from each search engine with the criteria string as input.  For each criteria for each search engine, the results were captured until a stopping criteria was met.  The search algorithm was performed as follows:

\begin{enumerate}
\item Remote call to search engine with current search start position and the current search criteria.
\item Parse results and extract paper title, authors and year of publication.
\item Compare results against stopping criteria.
	\begin{itemize}
	\item If stopping criteria met, stop search.
	\item If stopping criteria not met, increase search position by number of results and return to step 1.
	\end{itemize}
\end{enumerate}

The stopping criteria used was either after the first 1000 results, a limitation imposed by some of the search engines, or if ten consecutive results did not contain the search criteria phrase within the title.  After gathering all 12 data sets, the data was combined into a master list by systematically comparing the bibliographic information for each.  After producing a master list, a series of elimination rounds were performed to narrow the list of papers and identify primary sources.

\subsection{Elimination Rounds}

The elimination rounds were performed based on reading of the title, abstract and finally the paper itself.  While each elimination stage had a unique set of criteria for elimination the general procedure for elimination for the researchers was as follows.

\begin{itemize}
\item Each reviewer independently classified papers as relevant, irrelevant or uncertain based off elimination criteria
\item Those papers marked as relevant by both reviewers were kept and those marked irrelevant by both were thrown out.
\item Papers marked as relevant, or irrelevant by a single reviewer were combined with all papers marked as uncertain and discussed by both reviewers.  From this discussion, papers were either thrown out or kept until the next round of the review.  Ties were broken by an indepedent party.
\end{itemize}

\subsubsection{Title Elimination}

The first round of elimination was performed by strict examination of the title.  Each researcher was tasked with deciding on elimination by answering the following questions:

\begin{itemize}
\item Did the title contain a reference to 'role based access control' or 'RBAC'? 
\item Did the title contain a reference to 'model'?
\end{itemize}

\subsubsection{Abstract Elimination}

The second round of elimination was based on strict reading of the abstracts of papers that survived title elimination.  Researchers read each abstract and evaluated relevancy based off:

\begin{itemize}
\item Does the abstract mention a proposed model?
\item Does the abstract mention extension of role-based access control?
\item Does the abstract mention either an implementation, evaluation or domain for their model?
\end{itemize}

\subsubsection{Content Elimination}

The final elimination round involved taking the entire paper into consideration and answering five questions that would serve as the basis for elimination.  The data collected by answering these questions served as the basis towards answering the research questions.  The questions each reviewer attempted to answer based on the content of the paper was:

\begin{itemize}
\item Does this model extend the core model?
\item What reasons and evidence do researchers give that RBAC needs extension?
\item Was the paper and subsequent model inspired by a real world example?
\item Is there any evaluation of the proposed model? If yes, how did they do one? If no, why?
\item Did the authors implement their model?
\end{itemize}
