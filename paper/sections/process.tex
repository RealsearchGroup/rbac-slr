\section{Methodology and Process} \label{sec:process}

The systematic literature review process was developed ahead of time and agreed upon by the researchers following recommendations from Kitchenham's suggested processes \cite{kitchenham2007guidelines}.  The systematic literature review was performed in four stages:

\begin{itemize}
\item Development of a search strategy
\item Elimination of papers based on title
\item Elimination of papers based on abstract
\item Elimination of papers based on content and matching to elimination criteria
\end{itemize}

\subsection{Search Strategy}

For the first phase of our systematic literature review, an automated comprehensive search of multiple academic search engines was performed. The list of search engines were:

\begin{itemize}
\item Google Scholar - \url{http://scholar.google.com}
\item IEEExplore - \url{http://ieeexplore.ieee.org}
\item ACM Portal - \url{http://dl.acm.org}
\item CiteSeerX - \url{http://http://citeseerx.ist.psu.edu/index}
\end{itemize}

For each of the criteria below, a search was performed on each of the search engines for a total of 12 data sets.  The criteria used were:
\begin{itemize}
\item role based access control
\item RBAC
\item role-based access control
\end{itemize}

The search performed was done in an automated way using a set of scripts to query and collect data from each search engine with the criteria string as input.  For each criteria for each search engine, the results were captured until a stopping criteria was met.  Each run was performed as follows:

\begin{enumerate}
\item Remote call to search engine with current search start position and the current search criteria.
\item Parse results and extract paper title, authors and year of publication.
\item Compare results against stopping criteria.
\end{enumerate}

\begin{itemize}
\item If stopping criteria met, stop search.
\item If stopping criteria not met, increase search position by number of results and return to step 1.
\end{itemize}

The stopping criteria used was either after the first 1000 results, a limitation imposed by some of the search engines, or if ten consecutive results did not contain the search criteria phrase within the title.  After gathering all 12 data sets, the data was combined into a master list by systematically comparing the bibliographic information for each.  After producing a master list, a series of assessment rounds were performed to narrow the paper list and identify primary sources.

\subsection{Elimination Rounds}

The elimination rounds were performed based on reading of the title, abstract and finally the paper itself.  While each elimination stage had a unique set of criteria for elimination the general procedure for elimination for the researchers was as follows.

\begin{itemize}
\item Each reviewer independently classified papers as relevant, irrelevant or uncertain.
\item Those papers marked as relevant by both reviewers were kept and those marked irrelevant by both were thrown out.
\item Papers marked as relevant, or irrelevant by a single reviewer were combined with all papers marked as uncertain and discussed by both reviewers.  From this discussion, papers were either thrown out or kept until the next round of the review.  Ties were broken by an indepedent party.
\end{itemize}

The title elimination round was based off of whether role based access control and model were mentioned directly.

The second round of elimination was based off on reading the abstracts of the remaining papers.  The inclusion criteria for the abstract reading tried to answer the following questions:

\begin{itemize}
\item Does the abstract mention a proposed model?
\item Does the abstract mention extension of role-based access control?
\item Does the abstract mention either an implementation, evaluation or domain for their model?
\end{itemize}

The results of the searches is summarised below:

\begin{tabular}{|l|l|l|l|l|}
\hline
\textbf{Search Engines} & 
\textbf{RBAC} & 
\textbf{role based access control} & 
\textbf{role-based access control} & 
\textbf{Total}
\\\hline

Google Scholar & 651 & 213 & 435 & 1299
\\\hline
ACM Portal & 500 & 20 & 720 & 1240
\\\hline
IEEExplore & 200 & 40 & 230 & 470
\\\hline
CiteSeerX & 100 & 100 & 150 & 350
\\\hline
 &  &  &  & 
\\\hline
Totals & 1451 & 373 & 1535 & 3359
\\\hline

Combined &  &  &  & \textbf{1716}
\\\hline
\end{tabular}
