\section{Data Extraction} \label{sec:extraction}

After selection of primary sources, the next step was to extract data from each paper that pertained to our research questions in order to look for trends.  The first step was to take the individual data generated from the final elimination round and organize this information around the research questions.  During the paper reading round, and resulting data, the fact that the papers were logically falling into a number of categorizations became evident.  Thus, the first step undertaken was to answer the question of what categories exist for the RBAC model extensions and what papers feel into what categories.  The result of this paper categorization is displayed below:

\begin{center}
\begin{tabular}{|p{10cm}|p{3cm}|}
\hline
\textbf{Paper} & \textbf{Category}
\\\hline
Alam, M. and Hafner, M. and Breu, R. 2006 \cite{alam06:constraint} & Constraint \\\hline
Tzelepi, Sofia K. and Koukopoulos, Dimitrios K. and Pangalos, George 2001 \cite{tzelepi01:flexible} & Context \\\hline
Haibo, SHEN and Fan, HONG 2005 \cite{haibo05:context} & Context \\\hline
Damian G. Cholewka and Reinhardt A. Botha and Jan H. P. Eloff 2000 \cite{cholewka00:acontext-sensitive} & Context \\\hline
Huang, X. and Wang, H. and Chen, Z. and Lin, J. 2006 \cite{huang06:pervasive} & Context \\\hline
Motta, G.H.M.B. and Furuie, S.S. 2003 \cite{motta03:contextual} & Context \\\hline
Bao, Y. and Song, J. and Wang, D. and Shen, D. and Yu, G. 2008 \cite{bao08:role} & Context \\\hline
Yamazaki, W. and Hiraishi, H. and Mizoguchi, F. 2004 \cite{yamazaki104:designing} & Context \\\hline
Jian-min, H. and Xi-yu, L. and Hui-qun, Y. and Jun, T. 2008 \cite{han08:extended} & Context \\\hline
Thein, N. and others 2011 \cite{thein2011leveraging} & Context \\\hline
Zou, D. and He, L. and Jin, H. and Chen, X. 2009 \cite{zou2009crbac} & Context \\\hline
Hasebe, K. and Mabuchi, M. and Matsushita, A. 2010 \cite{hasebe10:capability} & Delegation \\\hline
Zhang, Z. and Zhang, X. and Sandhu, R. 2006 \cite{zhang06:collaborative} & Organizational \\\hline
Ni, Q. and Trombetta, A. and Bertino, E. and Lobo, J. 2007 \cite{ni2010privacy} & Privacy \\\hline
Masoumzadeh, A. and Joshi, J. 2008 \cite{masoumzadeh2008purbac} & Privacy \\\hline
Zhao, Y. and Zhao, Y. and Lu, H. 2008 \cite{zhao2008flexible} & Resource \\\hline
Bertino, E. and Catania, B. and Damiani, M.L. and Perlasca, P. 2005 \cite{damian2007geo} & Spatial \\\hline
Ray, I. and Kumar, M. and Yu, L. 2006 \cite{ray07:spatio} & Spatial \\\hline
Hansen, F. and Oleshchuk, V. 2003 \cite{hansen2003spatial} & Spatial \\\hline
Aich, S. and Sural, S. and Majumdar, A. 2007 \cite{aich07:STARBAC} & Spatio-Temporal \\\hline
Chen, L. and Crampton, J. 2008 \cite{chen08:spatio-temporal} & Spatio-Temporal \\\hline
Samuel, A. and Ghafoor, A. and Bertino, E. 2007 \cite{samuel07:spatio-temporal} & Spatio-Temporal \\\hline
Chandran, S. and Joshi, J. 2005 \cite{chandran05:llt} & Spatio-Temporal \\\hline
Ray, I. and Toahchoodee, M. 2007 \cite{ray07:spatio} & Spatio-Temporal \\\hline
Aich, S. and Mondal, S. and Sural, S. and Majumdar, A. 2009 \cite{aich09:role} & Spatio-Temporal \\\hline
Yao, L. and Kong, X. and Xu, Z. 2008 \cite{yao2008task} & Task \\\hline
ZHANG, S. and CHEN, X. and HOU, G. 2009 \cite{zhou2007team} & Task \\\hline
Oh, S. and Park, S. 2003 \cite{oh2003task} & Task \\\hline
Joshi, J.B.D. and Bertino, E. and Latif, U. and Ghafoor, A. 2005 \cite{joshi05:generalized} & Temporal \\\hline
\end{tabular}
\end{center}

Given that there were multiple papers for some categories, the researchers decided to tackle all further research questions by first analyzing the research question on a per category basis and then looking across all categories for generalization and trends.
