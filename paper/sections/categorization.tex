\section{Categorization}

After deciding upon a set of primary sources, the authors noticed that each paper fell into one or more categorization's of extensions to RBAC.
The categorizations were extracted by looking at how they are defined, what are their characteristics and how do individual categories relate
to other categories, and encapsulated by the following research question:

What categorizations exist within RBAC extensions? \\

\subsection{Definitions}

The primary sources contained either within the title, or by multiple references within the body of the text a reference to a quantifier for the
type of extension their model was.  For example, the paper "Privacy-aware role-based access control" \cite{ni2010privacy} contained "privacy" in
the title leading to the papers categorization of Privacy.  By comparison, the paper "An extended RBAC model based on granular logic" \cite{jian2008extended}
does not contain a direct categorization in the title or model name, but in reading the body of the paper the authors assessed that this paper was 
context based.  The authors offer definitions for each categorization and their characteristics based on the primary sources and the english definitions
for each.

\begin{itemize}

  \item Constraint:

  \item Context:

  \item Organizational:

  \item Privacy:

  \item Resource:

  \item Spatial:  Spatial constraints are location-based constraints in specifying access
	control policies. For example, in organizations, 	locations are enforced while a
	specific role is permitted to conduct an action. Consider that part-time employee works only in specific location.
	In such cases, the part-time employee role should access required resources only when the user is in the location. 
	Spatial constraints can incorporate either on roles, user-role assignments, or role-permission assignments. 

  \item Spatio-Temporal:

  \item Task: A task is a fundamental unit of a business activity. Different from core RBAC, in task-role-based access control model, roles are not directly associated with permissions. Roles are first associated with tasks, which are associated permissions. For example, the employee role is associated with a task, which is to write a report. Then, this task is associated with a permission.

  \item Temporal:  Temporal constraints are time-based constraints in specifying access
	control policies. For example, in organizations, 	periodic temporal durations are enforced while a
	specific role is permitted to conduct an action. Consider that part-time employee works only from 9:00 a.m. to 3:00 p.m.
	In such cases, the part-time employee role should access required resources during the interval. 
	Temporal constraints can incorporate either on roles, user-role assignments, or role-permission assignments.   
	
\end{itemize}

\subsection{Relationships}
