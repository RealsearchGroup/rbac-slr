\section{Core Role Based Access Control} \label{sec:core-rbac}

Since the basis for our review is extensions to the core model, we will describe the core model, associated entities and other terminology encountered across the space of our review.  The NIST RBAC model proposed by Ferraiolo et al. and later adopted as the official standard for RBAC by the International Committee for Information Technology Standards (INCITS) consists of four basic entities:

\begin{itemize}
\item a set of users \emph{Users}: A user can be a person or an agent.
\item  a set of roles \emph{Roles}: A role is a collection of permissions to perform a specific job function in an organization.
\item a set of permissions \emph{Permissions}: A permission refers to an access mode that can be exercised on an object in the system and a session relates a user to possibly many roles.
\item a set of sessions \emph{Sessions}: In each session, a user can be assigned to some of the roles, only when the corresponding role is enabled for activation for that time.		
\end{itemize}

In the RBAC, a user can exercise a permission only if the user are assigned to a role.
In addition to the four basic components, two functions are defined:
the user role assignment (UA) and the role
permission assignment (PA) functions.
UA models assignment of users to roles.
PA models assignment of permissions to roles.

%Each user incorporates a session    
%The user function maps each session to a
%single user, whereas the role function establishes a
%mapping between a session and a set of roles activated
%by the corresponding user in the session.

P1: Describe RBAC entities

P2: Note the 4 levels of RBAC

P3: Concisely describe level 1

P4: Concisely describe level 2

P5: COncisely describe level 3

P6: Concisely describe level 4

