\section{Role-Based Access Control Standard} \label{sec:core-rbac}

RBAC provides a high level abstraction of permissions management for operations, especially when sharing resources among roles within an organization. 
In cases where organizations were concerned with adopting RBAC, evaluation and comparison of RBAC technologies developed by different vendors was difficult.
To address this issue, NIST proposed a standard for RBAC.
NIST's RBAC standard can benefit organizations by lowering the cost of RBAC adoption~\cite{o20102010}.
The RBAC standard includes three components of RBAC: core RBAC, hierarchical RBAC, and constrained RBAC. 
Hierarchical RBAC and constrained RBAC are developed by incorporating new features into core RBAC.
Each component includes a corresponding RBAC Reference model.

We describe the four entities of these reference models.

\begin{itemize}
\setlength{\itemsep}{0.25pt}
\item \emph{Users}: A user is defined as a human being. Although the concept of a user can be extended to include machines, networks, or intelligent autonomous agents, the definition is limited to a person in the RBAC standard. 
\item \emph{Roles}: A role is a job function within the context of an organization with some associated semantics regarding the authority and responsibility conferred on the user assigned to the role.
\item \emph{Permissions}: A permission is an approval to perform an operation on one or more RBAC protected objects in the system.
\item \emph{Sessions}: A session is a mapping between a user and an activated subset of roles that are assigned to the user.
\end{itemize}

%\subsection{Core RBAC} 
%The four entities of the core RBAC Reference Model are:

In RBAC, a user can exercise a permission only if the user is assigned to a role that contains the permission.
In addition to the four basic entities, two functions are defined:
user assignment ($UA$) and permission assignment ($PA$) functions.
$UA$ represents assignment of users to roles.
$PA$ represents assignment of permissions to roles.
Permissions are associated with possible users' pre-defined operation on an object (e.g., execute a file).
Note that, at user or role activation, a session associated with user or role is established.

On the top of the core RBAC Reference Model, the hierarchical RBAC Reference Model adds role hierarchies ($RH$) as a feature. 
The role structure in an organization may use
a role $r_1$, which inherits all permissions of another role $r_2$.
Concept of the role inheritance describes the many-to-many mapping role inheritance relations among roles. Therefore, more than one role (e.g., two roles $r_1$ and $r_1'$) can inherit all permissions of $r_2$.

The constrained RBAC Reference Model adds separation of duty relations to the core and hierarchical RBAC Reference Models. Separation of duty relations enforce conflicts of interest among roles. 
The Constrained RBAC model defines two types of constraints placed on the user-role assignments:
Static Separation of Duty (SSoD) and Dynamic Separation of Duty (DSD).

\begin{itemize}
\item SSoD relations define constraints as a pair ($role set$, $n$) statically where no user is assigned to more than $n$ roles from the $role set$.
Suppose roles $Role_A$ and $Role_B$ in $role set$ conflict with each other and $n$ is 2. A user who is assinged to $Role_A$ cannot be assigned to $Role_B$.
%For example, an accounting clerk role requests a check and an accounting manager role approves the check. 
%These two roles must be mutually exclusive to avoid a situation where one approves the check that the other requested.
	
\item DSD relations define constraints as a pair ($role set$, $n$) dynamically where $n$ is a number, with the property that
user session may not activate more than $n$ roles from the $role set$.
For situations where multiple roles can be associated with a single user and $n$ is 2, a user session cannot be assigned to both $Role_A$ and $Role_B$ at the same time. 
%Consider that a user can be assigned to the accounting clerk and accounting manager role at the same time.
%In such a situation, DSD can enforce that the accounting manager role does not approve her/his requested checks but can only approve checks that others have requested.
\end{itemize}

