
% Template for Elsevier CRC journal article
% version 1.2 dated 09 May 2011

% This file (c) 2009-2011 Elsevier Ltd.  Modifications may be freely made,
% provided the edited file is saved under a different name

% This file contains modifications for Procedia Computer Science
% but may easily be adapted to other journals

% Changes since version 1.1
% - added "procedia" option compliant with ecrc.sty version 1.2a
%   (makes the layout approximately the same as the Word CRC template)
% - added example for generating copyright line in abstract

%-----------------------------------------------------------------------------------

%% This template uses the elsarticle.cls document class and the extension package ecrc.sty
%% For full documentation on usage of elsarticle.cls, consult the documentation "elsdoc.pdf"
%% Further resources available at http://www.elsevier.com/latex

%-----------------------------------------------------------------------------------

%%%%%%%%%%%%%%%%%%%%%%%%%%%%%%%%%%%%%%%%%%%%%%%%%%%%%%%%%%%%%%
%%%%%%%%%%%%%%%%%%%%%%%%%%%%%%%%%%%%%%%%%%%%%%%%%%%%%%%%%%%%%%
%%                                                          %%
%% Important note on usage                                  %%
%% -----------------------                                  %%
%% This file should normally be compiled with PDFLaTeX      %%
%% Using standard LaTeX should work but may produce clashes %%
%%                                                          %%
%%%%%%%%%%%%%%%%%%%%%%%%%%%%%%%%%%%%%%%%%%%%%%%%%%%%%%%%%%%%%%
%%%%%%%%%%%%%%%%%%%%%%%%%%%%%%%%%%%%%%%%%%%%%%%%%%%%%%%%%%%%%%

%% The '3p' and 'times' class options of elsarticle are used for Elsevier CRC
%% Add the 'procedia' option to approximate to the Word template
%\documentclass[3p,times,procedia]{elsarticle}
\documentclass[3p,times]{elsarticle}

%% The `ecrc' package must be called to make the CRC functionality available
\usepackage{ecrc}

%% The ecrc package defines commands needed for running heads and logos.
%% For running heads, you can set the journal name, the volume, the starting page and the authors

%% set the volume if you know. Otherwise `00'
\volume{00}

%% set the starting page if not 1
\firstpage{1}

%% Give the name of the journal
\journalname{Information and Software Technology}


%% Give the author list to appear in the running head
%% Example \runauth{C.V. Radhakrishnan et al.}
\runauth{}

%% The choice of journal logo is determined by the \jid and \jnltitlelogo commands.
%% A user-supplied logo with the name <\jid>logo.pdf will be inserted if present.
%% e.g. if \jid{yspmi} the system will look for a file yspmilogo.pdf
%% Otherwise the content of \jnltitlelogo will be set between horizontal lines as a default logo

%% Give the abbreviation of the Journal.  Contact the journal editorial office if in any doubt
\jid{procs}

%% Give a short journal name for the dummy logo (if needed)
\jnltitlelogo{IST}

%% Provide the copyright line to appear in the abstract
%% Usage:
%   \CopyrightLine[<text-before-year>]{<year>}{<restt-of-the-copyright-text>}
%   \CopyrightLine[Crown copyright]{2011}{Published by Elsevier Ltd.}
%   \CopyrightLine{2011}{Elsevier Ltd. All rights reserved}
\CopyrightLine{2011}{Published by Elsevier Ltd.}

%% Hereafter the template follows `elsarticle'.
%% For more details see the existing template files elsarticle-template-harv.tex and elsarticle-template-num.tex.

%% Elsevier CRC generally uses a numbered reference style
%% For this, the conventions of elsarticle-template-num.tex should be followed (included below)
%% If using BibTeX, use the style file elsarticle-num.bst

%% End of ecrc-specific commands
%%%%%%%%%%%%%%%%%%%%%%%%%%%%%%%%%%%%%%%%%%%%%%%%%%%%%%%%%%%%%%%%%%%%%%%%%%

%% The amssymb package provides various useful mathematical symbols
\usepackage{amssymb}
%% The amsthm package provides extended theorem environments
%% \usepackage{amsthm}

%% The lineno packages adds line numbers. Start line numbering with
%% \begin{linenumbers}, end it with \end{linenumbers}. Or switch it on
%% for the whole article with \linenumbers after \end{frontmatter}.
%% \usepackage{lineno}

%\usepackage{natbib}

%% natbib.sty is loaded by default. However, natbib options can be
%% provided with \biboptions{...} command. Following options are
%% valid:

%%   round  -  round parentheses are used (default)
%%   square -  square brackets are used   [option]
%%   curly  -  curly braces are used      {option}
%%   angle  -  angle brackets are used    <option>
%%   semicolon  -  multiple citations separated by semi-colon
%%   colon  - same as semicolon, an earlier confusion
%%   comma  -  separated by comma
%%   numbers-  selects numerical citations
%%   super  -  numerical citations as superscripts
%%   sort   -  sorts multiple citations according to order in ref. list
%%   sort&compress   -  like sort, but also compresses numerical citations
%%   compress - compresses without sorting
%%
%% \biboptions{comma,round}

% \biboptions{}

% if you have landscape tables
\usepackage[figuresright]{rotating}

% put your own definitions here:
%   \newcommand{\cZ}{\cal{Z}}
%   \newtheorem{def}{Definition}[section]
%   ...
%\input{macros}

% add words to TeX's hyphenation exception list
%\hyphenation{author another created financial paper re-commend-ed Post-Script}

% declarations for front matter

\begin{document}

\begin{frontmatter}

%% Title, authors and addresses

%% use the tnoteref command within \title for footnotes;
%% use the tnotetext command for the associated footnote;
%% use the fnref command within \author or \address for footnotes;
%% use the fntext command for the associated footnote;
%% use the corref command within \author for corresponding author footnotes;
%% use the cortext command for the associated footnote;
%% use the ead command for the email address,
%% and the form \ead[url] for the home page:
%%
%% \title{Title\tnoteref{label1}}
%% \tnotetext[label1]{}
%% \author{Name\corref{cor1}\fnref{label2}}
%% \ead{email address}
%% \ead[url]{home page}
%% \fntext[label2]{}
%% \cortext[cor1]{}
%% \address{Address\fnref{label3}}
%% \fntext[label3]{}

\dochead{}
%% Use \dochead if there is an article header, e.g. \dochead{Short communication}
%% \dochead can also be used to include a conference title, if directed by the editors
%% e.g. \dochead{17th International Conference on Dynamical Processes in Excited States of Solids}

\title{A Systematic Literature Review on Role-Based Access Control}

%% use optional labels to link authors explicitly to addresses:
%% \author[label1,label2]{<author name>}
%% \address[label1]{<address>}
%% \address[label2]{<address>}

\author{
}

\address{}

\begin{abstract}
%% Text of abstract


Role-based access control (RBAC) has become an increasingly popular access
control for various applications such as web applications and
database applications. RBAC restricts access to resources based on identity of subjects and/or groups
called roles. 
Since RBAC is introduced in 1998, researchers have proposed various extended models of RBAC.
For example, they define additional constraints among roles (e.g., temporal constraints or location constraints) or hierarchy relationship of roles.
Our goal of this work is to study extended models of RBAC and analyze their extended features and claimed research contributions
to find limitations of current RBAC models and what extent of extended features that can be used for future RBAC.
We conduct a systematic literature review by collecting and synthesizing relevant research
papers. We initially collect XXXX papers from various sources such as IEEE and ACM websites and selected 26 papers systemically.
We perform a comparative analysis to find relationships among extended models and RBAC.
\\\\[TBD: Result]
\\[TBD: Conclusion]

\end{abstract}

\begin{keyword}
%% keywords here, in the form: keyword \sep keyword

%% PACS codes here, in the form: \PACS code \sep code

%% MSC codes here, in the form: \MSC code \sep code
%% or \MSC[2008] code \sep code (2000 is the default)

\end{keyword}

\end{frontmatter}

%%
%% Start line numbering here if you want
%%
% \linenumbers

%% main text
%\section{}
%\label{}
\section{Introduction} \label{sec:introduction}

Why is the base model of RBAC extended by newer models?

P1: Note history of the RBAC model, statistics, and official status

P2: Mention extensions exist in response to developments and new domains over time

P3: Mention audience, why they should care, and why exploring extensions space is important in relation to the reason the standard was originally concieved

P4: Talk about what this work is going to do, goal, brief process

P5: List goals or contributions

P6: Outline paper and sections


Role based access control (RBAC) was first introduced in the 1990 when the National Instittute for Standards and Technology (NIST) requested that a unified standard be created.  proposed standard RBAC model \cite{ferraiolo}

Role-based access control (RBAC) models \cite{ferraiolo} became popular used to govern access to critical resources.  In an RBAC model, roles represent a group of users who are involved in a specific job function in an organization. RBAC assigns permissions of specific actions on resources to roles instead of individual users.  Therefore, in order to gain roles' permission on specific resources, users acquire appropriate roles first.

RBAC is a generalized access control approach used for various applications including web services, database applications, and healthcare applications.  RBAC has advantages in maintaining and managing organization's security policies.  For example, if a user is to access manager role's resources within a given organization, security policy administrators simply add the user to be associated with the manager role.

Standard RBAC model considers only role-user association and role hierarchy.
Since standard RBAC model has limitations such as specifying environmental constraints or context information
Researchers developed extended models of RBAC to overcome the limitations.
However, as researchers often develop their own specialized extended models of RBAC,
their research cannot be generalized or compared with other research work appropriately.
As a result, researchers could take time on reinventing the wheel.
But how do we, as a community, ensure that a metric is suitable and acceptable for its intended purpose?

The goal of this work is to synthesize available research results on extended models of RBAC. We analyze their extended features and claimed research contributions to find limitations of current RBAC models and what extent of extended features by
comparing with similar research work.
We conducted a systematic literature review (SLR) to evaluate and interpret all available research relevant to a particular research question or topic area of interest.

Our research give benefits to a community as follows:

\begin{itemize}
\item Our work summarizes current extended RBAC research work and its contributions. By synthesizing the current results, our work shows a roadmap of current extended RBAC research.
\item Our work guides a direction for a standard of extended RBAC. Understanding the categorization and the motivation of the existing research results helps decide a standard of extended RBAC.
\item Our work shows a criteria in comparison among research results.
\item Our work helps identify the research challenges in the ares of security policies and suggest a future extension of RBAC.
\end{itemize}

\section{Methodology and Process} \label{sec:process}

The systematic literature review process was developed ahead of time and agreed upon by the researchers following recommendations from Kitchenham's suggested processes \cite{kitchenham2007guidelines}.  The systematic literature review was performed in four stages:

\begin{itemize}
\item Development of a search strategy
\item Elimination of papers based on title
\item Elimination of papers based on abstract
\item Elimination of papers based on content and matching to elimination criteria
\end{itemize}

\subsection{Search Strategy}

For the first phase of our systematic literature review, an automated comprehensive search of multiple academic search engines was performed. The list of search engines were:

\begin{itemize}
\item Google Scholar - \url{http://scholar.google.com}
\item IEEExplore - \url{http://ieeexplore.ieee.org}
\item ACM Portal - \url{http://dl.acm.org}
\item CiteSeerX - \url{http://http://citeseerx.ist.psu.edu/index}
\end{itemize}

For each of the criteria below, a search was performed on each of the search engines for a total of 12 data sets.  The criteria used were:
\begin{itemize}
\item role based access control
\item RBAC
\item role-based access control
\end{itemize}

The search performed was done in an automated way using a set of scripts to query and collect data from each search engine with the criteria string as input.  For each criteria for each search engine, the results were captured until a stopping criteria was met.  Each run was performed as follows:

\begin{enumerate}
\item Remote call to search engine with current search start position and the current search criteria.
\item Parse results and extract paper title, authors and year of publication.
\item Compare results against stopping criteria.
\end{enumerate}

\begin{itemize}
\item If stopping criteria met, stop search.
\item If stopping criteria not met, increase search position by number of results and return to step 1.
\end{itemize}

The stopping criteria used was either after the first 1000 results, a limitation imposed by some of the search engines, or if ten consecutive results did not contain the search criteria phrase within the title.  After gathering all 12 data sets, the data was combined into a master list by systematically comparing the bibliographic information for each.  After producing a master list, a series of assessment rounds were performed to narrow the paper list and identify primary sources.

\subsection{Elimination Rounds}

The elimination rounds were performed based on reading of the title, abstract and finally the paper itself.  While each elimination stage had a unique set of criteria for elimination the general procedure for elimination for the researchers was as follows.

\begin{itemize}
\item Each reviewer independently classified papers as relevant, irrelevant or uncertain.
\item Those papers marked as relevant by both reviewers were kept and those marked irrelevant by both were thrown out.
\item Papers marked as relevant, or irrelevant by a single reviewer were combined with all papers marked as uncertain and discussed by both reviewers.  From this discussion, papers were either thrown out or kept until the next round of the review.  Ties were broken by an indepedent party.
\end{itemize}

The title elimination round was based off of whether role based access control and model were mentioned directly.

The second round of elimination was based off on reading the abstracts of the remaining papers.  The inclusion criteria for the abstract reading tried to answer the following questions:

\begin{itemize}
\item Does the abstract mention a proposed model?
\item Does the abstract mention extension of role-based access control?
\item Does the abstract mention either an implementation, evaluation or domain for their model?
\end{itemize}

The results of the searches is summarised below:

\begin{tabular}{|l|l|l|l|l|}
\hline
\textbf{Search Engines} & 
\textbf{RBAC} & 
\textbf{role based access control} & 
\textbf{role-based access control} & 
\textbf{Total}
\\\hline

Google Scholar & 651 & 213 & 435 & 1299
\\\hline
ACM Portal & 500 & 20 & 720 & 1240
\\\hline
IEEExplore & 200 & 40 & 230 & 470
\\\hline
CiteSeerX & 100 & 100 & 150 & 350
\\\hline
 &  &  &  & 
\\\hline
Totals & 1451 & 373 & 1535 & 3359
\\\hline

Combined &  &  &  & \textbf{1716}
\\\hline
\end{tabular}

%\input{related}

%% The Appendices part is started with the command \appendix;
%% appendix sections are then done as normal sections
%% \appendix

%% \section{}
%% \label{}

%% References
%%
%% Following citation commands can be used in the body text:
%% Usage of \cite is as follows:
%%   \cite{key}         ==>>  [#]
%%   \cite[chap. 2]{key} ==>> [#, chap. 2]
%%

%% References with BibTeX database:

\bibliographystyle{elsarticle-num}
%\bibliography{yangtse,yangtse2,policy}
\bibliography{policy}
%% Authors are advised to use a BibTeX database file for their reference list.
%% The provided style file elsarticle-num.bst formats references in the required Procedia style

%% For references without a BibTeX database:

% \begin{thebibliography}{00}

%% \bibitem must have the following form:
%%   \bibitem{key}...
%%

% \bibitem{}

% \end{thebibliography}

\end{document}

%%
%% End of file `ecrc-template.tex'. 