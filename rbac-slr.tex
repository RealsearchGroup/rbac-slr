
% Template for Elsevier CRC journal article
% version 1.2 dated 09 May 2011

% This file (c) 2009-2011 Elsevier Ltd.  Modifications may be freely made,
% provided the edited file is saved under a different name

% This file contains modifications for Procedia Computer Science
% but may easily be adapted to other journals

% Changes since version 1.1
% - added "procedia" option compliant with ecrc.sty version 1.2a
%   (makes the layout approximately the same as the Word CRC template)
% - added example for generating copyright line in abstract

%-----------------------------------------------------------------------------------

%% This template uses the elsarticle.cls document class and the extension package ecrc.sty
%% For full documentation on usage of elsarticle.cls, consult the documentation "elsdoc.pdf"
%% Further resources available at http://www.elsevier.com/latex

%-----------------------------------------------------------------------------------

%%%%%%%%%%%%%%%%%%%%%%%%%%%%%%%%%%%%%%%%%%%%%%%%%%%%%%%%%%%%%%
%%%%%%%%%%%%%%%%%%%%%%%%%%%%%%%%%%%%%%%%%%%%%%%%%%%%%%%%%%%%%%
%%                                                          %%
%% Important note on usage                                  %%
%% -----------------------                                  %%
%% This file should normally be compiled with PDFLaTeX      %%
%% Using standard LaTeX should work but may produce clashes %%
%%                                                          %%
%%%%%%%%%%%%%%%%%%%%%%%%%%%%%%%%%%%%%%%%%%%%%%%%%%%%%%%%%%%%%%
%%%%%%%%%%%%%%%%%%%%%%%%%%%%%%%%%%%%%%%%%%%%%%%%%%%%%%%%%%%%%%

%% The '3p' and 'times' class options of elsarticle are used for Elsevier CRC
%% Add the 'procedia' option to approximate to the Word template
%\documentclass[3p,times,procedia]{elsarticle}
\documentclass[3p,times]{elsarticle}

%% The `ecrc' package must be called to make the CRC functionality available
\usepackage{ecrc}

%% The ecrc package defines commands needed for running heads and logos.
%% For running heads, you can set the journal name, the volume, the starting page and the authors

%% set the volume if you know. Otherwise `00'
\volume{00}

%% set the starting page if not 1
\firstpage{1}

%% Give the name of the journal
\journalname{Information and Software Technology}
\usepackage{times}
\usepackage{graphicx}
\usepackage{epsf}
\usepackage{verbatim}
%\usepackage{psfig}
\usepackage{cite}
\usepackage{url}
\usepackage{color}
\usepackage{alltt}


\newcommand{\Add}{\CodeIn{add}}
\newcommand{\AVTree}{\CodeIn{AVTree}}
\newcommand{\Assignment}[3]{$\langle$ \Object{#1}, \Object{#2}, \Object{#3} $\rangle$}
\newcommand{\BinaryTreeRemove}{\CodeIn{BinaryTree\_remove}}
\newcommand{\BinaryTree}{\CodeIn{BinaryTree}}
\newcommand{\Caption}{\vspace{-3ex}\caption}
\newcommand{\Char}[1]{`#1'}
\newcommand{\CheckRep}{\CodeIn{checkRep}}
\newcommand{\ClassC}{\CodeIn{C}}
\newcommand{\CodeIn}[1]{{\small\texttt{#1}}}
\newcommand{\CodeOutSize}{\scriptsize}
\newcommand{\Comment}[1]{}
\newcommand{\Ensures}{\CodeIn{ensures}}
\newcommand{\ExtractMax}{\CodeIn{extractMax}}
\newcommand{\FAL}{field-ordering}
\newcommand{\FALs}{field-orderings}
\newcommand{\Fact}{observation}
\newcommand{\Fix}[1]{{\large\textbf{FIX}}#1{\large\textbf{FIX}}}
\newcommand{\Get}{\CodeIn{get}}
\newcommand{\HashSet}{\CodeIn{HashSet}}
\newcommand{\HeapArray}{\CodeIn{HeapArray}}
\newcommand{\Intro}[1]{\emph{#1}}
\newcommand{\Invariant}{\CodeIn{invariant}}
\newcommand{\JUC}{\CodeIn{java.\-util.\-Collections}}
\newcommand{\JUS}{\CodeIn{java.\-util.\-Set}}
\newcommand{\JUTM}{\CodeIn{java.\-util.\-TreeMap}}
\newcommand{\JUTS}{\CodeIn{java.\-util.\-TreeSet}}
\newcommand{\JUV}{\CodeIn{java.\-util.\-Vector}}
\newcommand{\JMLPlusJUnit}{JML+JUnit}
\newcommand{\Korat}{Korat}
\newcommand{\Left}{\CodeIn{left}}
%\newcommand{\LinkedList}{\CodeIn{LinkedList}}
\newcommand{\Lookup}{\CodeIn{lookup}}
\newcommand{\MethM}{\CodeIn{m}}
\newcommand{\Node}[1]{\CodeIn{N}$_#1$}
\newcommand{\Null}{\CodeIn{null}}
\newcommand{\Object}[1]{\CodeIn{o}\ensuremath{_#1}}
\newcommand{\PostM}{\MethM$_{post}$}
\newcommand{\PreM}{\MethM$_{pre}$}
\newcommand{\Put}{\CodeIn{put}}
\newcommand{\Remove}{\CodeIn{remove}}
\newcommand{\RepOk}{\CodeIn{repOk}}
\newcommand{\Requires}{\CodeIn{requires}}
\newcommand{\Reverse}{\CodeIn{reverse}}
\newcommand{\Right}{\CodeIn{right}}
\newcommand{\Root}{\CodeIn{root}}
\newcommand{\Set}{\CodeIn{set}}
\newcommand{\State}[1]{2^{#1}}
\newcommand{\TestEra}{TestEra}
\newcommand{\TreeMap}{\CodeIn{TreeMap}}

\newenvironment{CodeOut}{\begin{scriptsize}}{\end{scriptsize}}
\newenvironment{SmallOut}{\begin{small}}{\end{small}}

\newcommand{\FixTao}[1]{{\large\textbf{FIXTAO}}#1{\large\textbf{FIXTAO}}}
\newcommand{\CommentTao}[1]{{\large\textbf{COMMENTTAO}}#1{\large\textbf{COMMENTTAO}}}

\newcommand{\pairwiseEquals}{PairwiseEquals}
\newcommand{\monitorEquals}{MonitorEquals}
\newcommand{\monitorWField}{WholeStateW}
\newcommand{\traverseField}{WholeState}
\newcommand{\monitorSMSeq}{ModifyingSeq}
\newcommand{\monitorSeq}{WholeSeq}

\newcommand{\IntStack}{\CodeIn{UBStack}}
\newcommand{\UBStack}{\CodeIn{UBStack}}
\newcommand{\BSet}{\CodeIn{BSet}}
\newcommand{\BBag}{\CodeIn{BBag}}
\newcommand{\ShoppingCart}{\CodeIn{ShoppingCart}}
\newcommand{\BankAccount}{\CodeIn{BankAccount}}
\newcommand{\BinarySearchTree}{\CodeIn{BinarySearchTree}}
\newcommand{\LinkedList}{\CodeIn{LinkedList}}

\newcommand{\Book}{\CodeIn{Book}}
\newcommand{\Library}{\CodeIn{Library}}

\newcommand{\Jtest}{Jtest}
\newcommand{\JCrasher}{JCrasher}
\newcommand{\Daikon}{Daikon}
\newcommand{\JUnit}{JUnit}

\newcommand{\trie}{trie}

\newcommand{\Perl}{Perl}

\newcommand{\Equals}{\CodeIn{equals}}
\newcommand{\Pairwise}{PairwiseEquals}
\newcommand{\Subgraph}{MonitorEquals}
\newcommand{\Concrete}{WholeState}
\newcommand{\ModSeq}{ModifyingSeq}
\newcommand{\Seq}{WholeSeq}
\newcommand{\Aeq}{equality}

\newcommand{\Pair}[2]{\ensuremath{\langle #1, #2 \rangle}}
\newcommand{\Triple}[3]{\ensuremath{\langle #1, #2, #3 \rangle}}
\newcommand{\SetSuch}[2]{\ensuremath{\{ #1 | #2 \}}}
%\newtheorem{definition}{Definition}
%\newtheorem{theorem}[definition]{Theorem}
\newcommand{\Equiv}[2]{\ensuremath{#1 \EquivSTRel{} #2}}
\newcommand{\EquivME}{\Equiv}
\newcommand{\EquivST}{\Equiv}
\newcommand{\EquivSTRel}{\ensuremath{\cong}}
\newcommand{\Redundant}[2]{\ensuremath{#1 \lhd #2}}
\newcommand{\VB}{\ensuremath{\mid}}
\newcommand{\MES}{method-entry state}
\newcommand{\SmallSpace}{\vspace*{-1.5ex}}
\newcommand{\Item}{\SmallSpace\item}
\newenvironment{Itemize}{\begin{itemize}}{\end{itemize}\SmallSpace}
\newenvironment{Enumerate}{\begin{enumerate}}{\end{enumerate}\SmallSpace}

%\newcommand{\ImprovementRatio}{20\%}
\newcommand{\SubjectCount}{eleven}
\newcommand{\DSSubjectCount}{two}

\newcommand{\CenterCell}[1]{\multicolumn{1}{c|}{#1}}


%% Give the author list to appear in the running head
%% Example \runauth{C.V. Radhakrishnan et al.}
\runauth{}

%% The choice of journal logo is determined by the \jid and \jnltitlelogo commands.
%% A user-supplied logo with the name <\jid>logo.pdf will be inserted if present.
%% e.g. if \jid{yspmi} the system will look for a file yspmilogo.pdf
%% Otherwise the content of \jnltitlelogo will be set between horizontal lines as a default logo

%% Give the abbreviation of the Journal.  Contact the journal editorial office if in any doubt
\jid{procs}

%% Give a short journal name for the dummy logo (if needed)
\jnltitlelogo{IST}

%% Provide the copyright line to appear in the abstract
%% Usage:
%   \CopyrightLine[<text-before-year>]{<year>}{<restt-of-the-copyright-text>}
%   \CopyrightLine[Crown copyright]{2011}{Published by Elsevier Ltd.}
%   \CopyrightLine{2011}{Elsevier Ltd. All rights reserved}
\CopyrightLine{2011}{Published by Elsevier Ltd.}

%% Hereafter the template follows `elsarticle'.
%% For more details see the existing template files elsarticle-template-harv.tex and elsarticle-template-num.tex.

%% Elsevier CRC generally uses a numbered reference style
%% For this, the conventions of elsarticle-template-num.tex should be followed (included below)
%% If using BibTeX, use the style file elsarticle-num.bst

%% End of ecrc-specific commands
%%%%%%%%%%%%%%%%%%%%%%%%%%%%%%%%%%%%%%%%%%%%%%%%%%%%%%%%%%%%%%%%%%%%%%%%%%

%% The amssymb package provides various useful mathematical symbols
\usepackage{amssymb}
%% The amsthm package provides extended theorem environments
%% \usepackage{amsthm}

%% The lineno packages adds line numbers. Start line numbering with
%% \begin{linenumbers}, end it with \end{linenumbers}. Or switch it on
%% for the whole article with \linenumbers after \end{frontmatter}.
%% \usepackage{lineno}

%% natbib.sty is loaded by default. However, natbib options can be
%% provided with \biboptions{...} command. Following options are
%% valid:

%%   round  -  round parentheses are used (default)
%%   square -  square brackets are used   [option]
%%   curly  -  curly braces are used      {option}
%%   angle  -  angle brackets are used    <option>
%%   semicolon  -  multiple citations separated by semi-colon
%%   colon  - same as semicolon, an earlier confusion
%%   comma  -  separated by comma
%%   numbers-  selects numerical citations
%%   super  -  numerical citations as superscripts
%%   sort   -  sorts multiple citations according to order in ref. list
%%   sort&compress   -  like sort, but also compresses numerical citations
%%   compress - compresses without sorting
%%
%% \biboptions{comma,round}

% \biboptions{}

% if you have landscape tables
\usepackage[figuresright]{rotating}

% put your own definitions here:
%   \newcommand{\cZ}{\cal{Z}}
%   \newtheorem{def}{Definition}[section]
%   ...

% add words to TeX's hyphenation exception list
%\hyphenation{author another created financial paper re-commend-ed Post-Script}

% declarations for front matter

\begin{document}

\begin{frontmatter}

%% Title, authors and addresses

%% use the tnoteref command within \title for footnotes;
%% use the tnotetext command for the associated footnote;
%% use the fnref command within \author or \address for footnotes;
%% use the fntext command for the associated footnote;
%% use the corref command within \author for corresponding author footnotes;
%% use the cortext command for the associated footnote;
%% use the ead command for the email address,
%% and the form \ead[url] for the home page:
%%
%% \title{Title\tnoteref{label1}}
%% \tnotetext[label1]{}
%% \author{Name\corref{cor1}\fnref{label2}}
%% \ead{email address}
%% \ead[url]{home page}
%% \fntext[label2]{}
%% \cortext[cor1]{}
%% \address{Address\fnref{label3}}
%% \fntext[label3]{}

\dochead{}
%% Use \dochead if there is an article header, e.g. \dochead{Short communication}
%% \dochead can also be used to include a conference title, if directed by the editors
%% e.g. \dochead{17th International Conference on Dynamical Processes in Excited States of Solids}

\title{A Systematic Literature Review of Modeling, Specifying, and Analysis of Access Control Policies}

%% use optional labels to link authors explicitly to addresses:
%% \author[label1,label2]{<author name>}
%% \address[label1]{<address>}
%% \address[label2]{<address>}

\author{
}

\address{}

\begin{abstract}
%% Text of abstract
Abstract.
\end{abstract}

\begin{keyword}
%% keywords here, in the form: keyword \sep keyword

%% PACS codes here, in the form: \PACS code \sep code

%% MSC codes here, in the form: \MSC code \sep code
%% or \MSC[2008] code \sep code (2000 is the default)

\end{keyword}

\end{frontmatter}

%%
%% Start line numbering here if you want
%%
% \linenumbers

%% main text
%\section{}
%\label{}
\section{Introduction} \label{sec:introduction}

Role based access control (RBAC) is used for maintaining and managing an organization's 
access control based on assigning permissions to roles, and roles to users, instead of 
assigning the permissions directly to individual users. 
The use of an RBAC model has become popular since the National Institute for Standards and Technology (NIST) 
first proposed the RBAC standard model in the 1990s \cite{o20102010}. The RBAC standard model is widely used for securing 
various applications including web services, database applications, and healthcare applications. 
In the RBAC standard model, roles represent a set of permissions needed to perform a particular job function within an organization.  
One to many users who are involved in that specific job function within the organization can then be 
assigned to the role to inherit the required access instead of assigning individual permissions to individual users. 
The ability to logically group both users and permissions into smaller total entities becomes paramount for managing access control
as an organization grows and the number of permissions and users scales upward, 
The RBAC standard model has advantages for managing access control. For example, if a user requires access to resources 
associated with the manager role within a given organization, security policy administrators need only associate the user with the manager role.

The RBAC standard was first proposed in the 1990s when NIST 
requested that a unified standard be created by combining the Ferraiolo and Kuhn model \cite{ferraiolokuhn} with the framework 
proposed by Sandhu, et al \cite{sandhu1996role}.  
In 2004, this standard was adopted as ANSI/INCITS 359-2004 approved by American National Standards Institute (ANSI) and the InterNational Committee for Information Technology Standards (INCITS). 
The development of a standard was inspired by an economic impact study done during the 1990s and later confirmed in 2010 \footnote{\url{http://csrc.nist.gov/groups/SNS/rbac/documents/20101219_RBAC2_Final_Report.pdf}} that showed the cost savings of RBAC implementation and maintenance. 
Prior to the development of the RBAC standard, vendors proposed and implemented their own defined RBAC features without general agreement on a unified definition of RBAC or feature set. 
The RBAC standard addressed this issue by providing definitions and a functional specification of RBAC features using a reference model.

Innovation and the use of RBAC in new domains has led to scenarios that, by some accounts, the standard RBAC model cannot handle \cite{kuhn2010adding}.  
For example, Ni, Q. et al \cite{ni2010privacy} state that RBAC is ``not designed to enforce privacy policies and barely 
meet privacy protection requirements'' with the introduction of privacy concerns in domains such as healthcare. 
Since the introduction of the NIST RBAC standard in 2000, extension models to the RBAC standard have appeared in the literature, each adding one or more features on top of the core RBAC model. 
A cursory examination of the extension model papers does not provide good comparisons for developers and researchers to build upon. 
Further, these extension models are each building upon and adding features to a standard that was designed to reduce the economic impact experienced by enterprises and increase interoperability \cite{o20102010}.

\textit{The goal of our work is to aid practitioners and researchers in choosing an extension to the RBAC standard model, and in understanding
how extensions to RBAC are evaluated by providing an assessment of the state of extensions to the RBAC standard model. We accomplished this through: 
establishment of a set of extension categories, an examination of the state of the art in evaluations of extension models and a breakdown of the 
motivations that have led to extension of the RBAC standard model.} With respect to our goal, the authors addressed the following research questions:

\begin{itemize}
\setlength{\itemsep}{0.25pt}
\item RQ1: What categories exist within extensions to RBAC?
\item RQ2: What are the motivations behind extensions to RBAC?
\item RQ3: Do the extensions to RBAC have corresponding implementations?
\item RQ4: How are extensions to RBAC evaluated theoretically and in practice?
\item RQ5: What domains have extensions to RBAC been created for?
\item RQ6: What commonalities or generalizations exist across all categories?
\end{itemize}

The authors performed a systematic literature review designed to explore the current body of research in the area of extensions to the RBAC standard model. 
The review we performed yielded 1716 papers, of which 28 were deemed primary sources for inclusion as extension models to the RBAC standard model.  
This review is intended to serve as a starting place for researcher's looking to tackle new problems in the realm of authorization and a reference guide for discovering
the current state of extensions to the RBAC standard model. 
For developer's looking to find a model to fit their potentially unique access control needs, this review will provide a basis for comparison and easy look-up for what extension model to use.  
Further, the review provides insight into the state of the art in evaluating RBAC extension models.

Our research provides contributions to the community in the following ways:

\begin{itemize}
\setlength{\itemsep}{0.25pt}
\item Summarizes current research on extensions to RBAC.
\item Guides a direction for a new standard based on an extension to the RBAC standard model.
\end{itemize}

The rest of the paper is organized as follows. 
Section 2 covers background and the RBAC standard. 
Section 3 lays out the process used in conducting the review. 
Sections 4-9 present the analysis and discussion of the research questions. 
Section 10 is a general discussion section brings together all the research questions.
Section 11 contains the final conclusions.

\section{Methodology and Process} \label{sec:process}

We adopted and applied a systematic literature review process following recommendations from Kitchenham's suggested processes~\cite{kitchenham2007guidelines}.  The systematic literature review process was broken down into four stages and the rest of this section is broken down by each stage.  The stages were as follows:

\begin{itemize}
\setlength{\itemsep}{0.25pt}
\item Step 1: Development of a search strategy
\item Step 2: Elimination of papers based on title criteria
\item Step 3: Elimination of papers based on abstract criteria
\item Step 4: Elimination of papers based on content and elimination criteria
\end{itemize}

\subsection{Step 1: Search Strategy}

For the first phase of our systematic literature review, we developed a search strategy for finding papers.  The search strategy was executed by an automated comprehensive search taking as input a set of academic search engines and a list of search terms.  The search was performed by applying each search term to each engine incrementally until the stopping criteria were met. Table \ref{tab:search_results} lists the four search engines along the left most column with the three search terms across the top row along with the papers for each criterion and engine combination.  The search algorithm was performed as follows:

\begin{enumerate}
\setlength{\itemsep}{0.25pt}
\item Call to search engine with current search position and current search term.
\item Parse results and extract paper title, authors and year of publication.
\item Compare results against stopping criteria:
	\begin{itemize}
	\item If the size of the result set is greater than or equal to 1000 then stop.
	\item If the last ten results did not contain the search term phrase within the title then stop.
	\end{itemize}
\item If stopping criteria not met, increment search position and go back to step one.
\end{enumerate}

The result set size stopping criteria was chosen due to a technical limitation of some search engines.  The stopping criteria related to the last ten titles are meant to stop after relevant results are no longer being returned by the search engine.  After gathering all 12 data sets, we combined the papers into a master list, which includes only distinct papers by systematically comparing the bibliographic information for each.
Out of the master list of 1,716 papers, two reviewers, denoted by Reviewer 1 and Reviewer 2, conducted a series of elimination rounds to narrow the list of papers and identify primary sources. Table \ref{tab:eliminations} shows the total number of papers selected by each reviewer for each round and how many papers from the disjoint set for each round survived to the next round.

\begin{table}
\centering
\caption{Paper counts after applying search strategy}
\vspace{0.1 in}
\begin{tabular}{ | l | r | r | r | r | }

\hline
 & 
\textbf{RBAC} & 
\textbf{role based access control} & 
\textbf{role-based access control} & 
\textbf{Total}
\\\hline

Google Scholar & 651 & 213 & 435 & 1299 \\\hline
ACM Portal & 500 & 20 & 720 & 1240 \\\hline
IEEExplore & 200 & 40 & 230 & 470 \\\hline
CiteSeerX & 100 & 100 & 150 & 350 \\\hline
 &  &  &  & \\\hline
Totals & 1451 & 373 & 1535 & 3359 \\\hline
Combined &  &  &  & \textbf{1716} \\\hline

\end{tabular}
\label{tab:search_results}
\end{table}

\subsection{Steps 2-4: Elimination Rounds}

The elimination rounds were conducted based on reading of the title, abstract, and finally the paper itself.  While each elimination stage had a unique set of criteria for elimination, the general procedure for elimination for the researchers was as follows.

\begin{itemize}
\setlength{\itemsep}{0.25pt}
\item The two first authors independently classified papers as relevant, irrelevant or uncertain based on elimination criteria
\item Those papers marked as relevant by both reviewers were kept and those marked irrelevant by both were thrown out.
\item Papers marked as relevant or irrelevant by a single reviewer were combined with all papers marked as uncertain and discussed by both reviewers.  From this discussion, papers were either thrown out or kept until the next round of the review.
\end{itemize}

\subsubsection{Step 2: Title Elimination}

The first round of elimination was performed by examination based on the title.  Each author was tasked with deciding on elimination by answering the following questions:

\begin{itemize}
\setlength{\itemsep}{0.25pt}
\item Did the title contain a reference to 'role-based access control' or 'RBAC'? 
\item Did the title contain a reference to 'model'?
\end{itemize}

The title elimination round resulted in Reviewer 1 selecting 305 papers, and Reviewer 2 selecting 176 papers with 149 papers of overlap between the two. 
There were 332 papers found not to be in common, of which, 149 were rejected and 141 retained after a second review.

\subsubsection{Step 3: Abstract Elimination}

The second round of elimination was based on reading of the abstracts of papers that survived title elimination.  Researchers read each abstract and evaluated relevancy based off:

\begin{itemize}
\setlength{\itemsep}{0.25pt}
\item Does the abstract mention a proposed model?
\item Does the abstract mention extension of role-based access control?
\item Does the abstract mention an implementation, evaluation, or domain for their model?
\end{itemize}

The abstract elimination round resulted in Reviewer 1 selecting 86 papers, and Reviewer 2 selecting 102 papers with 51 papers of overlap between the two. 
There were 137 papers found not to be in common, of which, 116 were rejected and 21 retained after a second review.

\subsubsection{Step 4: Content Elimination}

The final elimination round involved reading the entire paper and answering five questions that would serve as the basis for elimination.  The data collected by answering these questions served as the basis towards answering the research questions.  Each reviewer seeks to answer following questions based on the content of the paper:

\begin{enumerate}
\setlength{\itemsep}{0.25pt}
\item Does this model extend the RBAC Reference Model (Exclusion)
\item Do the researchers give evidence that the RBAC Reference Model needs extension? (Inclusion)
\item Was the paper and subsequent model inspired by a real world example?  (Conditional Inclusion)
\item Did the researchers offer any evaluation of the proposed model? If yes, how did they do one? If no, why? (Conditional Inclusion)
\item Did the authors implement their model? (Inclusion)
\end{enumerate}

Question 1 was a definitive exclusion criterion as any paper that failed in the affirmative was rejected.  Questions 3 and 4 were marked as conditional includes given that they were connected in making a decision. 
A paper that met question 3 but not 4, or met 4 but not 3 would be included because for some cases the real world examples served as research evaluations and without this conditional include the paper list size would be too small to be significant.

The content elimination round resulted in Reviewer 1 selecting 46 papers, and Reviewer 2 selecting 42 papers with 24 papers of overlap between the two. 
Between the two reviewers selections, there were 64 papers not in common, of which, 59 were rejected and 5 retained after a second review.

\begin{table}
\centering
\caption{Elimination Rounds}
\vspace{0.1 in}
\begin{tabular}{ l l | r | r | r | }
\cline{3-5}

 & \multicolumn{1}{ c| }{} & \textbf{Title} & \textbf{Abstract} & \textbf{Content} \\ \hline

\multicolumn{1}{ |c| }{Reviewer 1}  &  & 305 & 86 & 46 \\ \hline
\multicolumn{1}{ |c| }{Reviewer 2} &  & 176 & 102 & 42 \\ \hline

\multicolumn{1}{ c| }{} & Overlap & 149 & 51 & 24 \\ \cline{2-5}
\multicolumn{1}{ c| }{} & Disjoint & 332 & 137 & 64 \\ \cline{2-5}
\multicolumn{1}{ c| }{} & Rejected & 191 & 116 & 61 \\ \cline{2-5}
\multicolumn{1}{ c| }{} & Retained & 141 & 21 & 5 \\ \cline{2-5}
\multicolumn{1}{ c| }{} & \textbf{Num Left} & 290 & 72 & 27 \\ \cline{2-5}

\end{tabular}
\label{tab:eliminations}
\end{table}


\subsection{Extraction}

After selection of primary sources, the next step was to extract data from each paper that pertained to our research questions in order to look for trends. 
The first step was to take the individual data generated from the final elimination round and organize this information around the research questions. 
During the paper reading round and resulting data, the fact that the papers were falling into a number of categorizations became evident. 
Thus, the first step undertaken was to answer the question of what categories exist for the RBAC extension models and what papers fell into what categories.


%% The Appendices part is started with the command \appendix;
%% appendix sections are then done as normal sections
%% \appendix

%% \section{}
%% \label{}

%% References
%%
%% Following citation commands can be used in the body text:
%% Usage of \cite is as follows:
%%   \cite{key}         ==>>  [#]
%%   \cite[chap. 2]{key} ==>> [#, chap. 2]
%%

%% References with BibTeX database:

\bibliographystyle{elsarticle-num}
\bibliography{yangtse}

%% Authors are advised to use a BibTeX database file for their reference list.
%% The provided style file elsarticle-num.bst formats references in the required Procedia style

%% For references without a BibTeX database:

% \begin{thebibliography}{00}

%% \bibitem must have the following form:
%%   \bibitem{key}...
%%

% \bibitem{}

% \end{thebibliography}

\end{document}

%%
%% End of file `ecrc-template.tex'. 